\section*{謝辞}

本研究を進めるにあたって,多くの方々におせわになりました.
まず,1 年間(正確にはレポートが完成するまでの15 ヶ月間)実験の方法や解析などを指導してくださった中家さんと隅田さんに感謝します.ほとんどのメンバーが今後も大学院でお世話になりますがよろしくおねがいします.また,課題研究P2 の理論パートを担当してくださった畑さんにも同様に感謝します.理論パートで扱った内容が実験に直結することもありスムーズな実験背景の理解の助けとなりました.磁場装置の開発について助言をしてくださり,また磁場測定装置を貸してくださった化研の岩下さんにも感謝します.実験で用いた磁場装置の完成は岩下さんなしではありえませんでした.

今回の実験を行うにあたってKEK の加速器科学インターンシップを用いましたが,このような機会を与えてくださった高エネルギー加速器科学研究奨励会の\CID{8705}崎さん,三國さん,KEK およびJ-PARC の関係者の方々,さらにKEK の三宅さん,三部さん,大谷さん,パラサイト実験をさせていただいた$g-2$ Beam Profiling Monitor Group のみなさんに感謝します.学部生ではめったに得られない貴重な経験を今後に生かして行きたいと思います.

また,TA としてP2 全体をサポートしてくださった赤塚さん,関さん,野口さん,安留さんにも感謝します.理論・実験両ゼミでの助言の数々や本実験でのご支援がとてもありがたかったです.その他,高エネ研究室の院生の皆さんにはP2 部屋を覗きに来ていただいたり,機材発送の支援などをしていただきました.1 月に行われた同志社大学文学部との交流会では実験直前に異分野との交流を通して一息つくことができました.その他,多くの方々のお世話になり,今回の実験を行うことができました.みなさま,ありがとうございました.