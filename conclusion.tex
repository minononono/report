\section{結論}
% 結論といいながら考察も含む
全ての測定の解析の結果をまとめると図~\ref{tab:result_conclusion}にようになる.
いずれの測定結果においても,理論値と誤差の範囲で一致した結果が得られた.
ここでは,物理的なそれぞれの意味について確認し,今後の展望について考察していきたい.
また,2つの検出器の測定結果についての違いも考察する.

\begin{table}
  \centering
  \caption{測定結果まとめ}
  \label{tab:result_conclusion}
  \begin{tabular}{ccccc}\toprule
    & & 理論値 & P.S.検出器 & NaI検出器\\ \midrule
    \multicolumn{2}{c}{寿命($\mathrm{ns}$)} &  2197 & $2216 \pm 54^{ + 35}_{ - 15}$ & $2184\pm52$ \\
    \multicolumn{2}{c}{$g$因子}               & 2.002 & $2.010 \pm 0.011^{+0.008}_{-0.017}$ &  \\ % NaIの解析結果を追記する  
                         & $\rho$           &  0.75 & - & $0.66\pm0.02\pm0.14$ \\
    ミッシェルパラメータ & $\delta$         &  0.75 & - & $0.61\pm0.11\pm0.18$ \\
                         & $\xi$            &  1    & - & $0.983\pm0.017$\\ \bottomrule
  \end{tabular}
\end{table}

\subsection{寿命の測定}
寿命の大きさを主に決定するのは相互作用の種類であり,その結合が強ければ寿命は短くなることが分かる.
今回のミューオン崩壊は弱い相互作用によって生じており,その名の通り結合が弱くFermi結合定数が小さい.
そのため寿命は比較的長く,測定が可能だった.
実際の測定値も結合定数から予想される値と一致し,弱い相互作用のその結合の弱さを検証できたと言える.
寿命の測定は基本的な定数である結合定数の測定として重要な手段であるためより高精度の測定は重要である.
今回の測定では,統計誤差が大きいため統計を貯めることが目標である.

時間の測定が単純に影響がでる寿命測定では2つの観点でPS検出器はNaI検出器より良い測定結果が得られることが予想された.
1つはプラスチックはNaIに比べシンチレータ信号の立ち上がり時間が短いため,NaI検出器に比べ時間分解能が良くなり誤差が減るということである.
もう1つがプラスチックはNaIに比べシンチレータ信号は立ち下がり時間が短いため,ハイレートの測定が可能であり統計量を増やしやすいということことである.
一方,実際の測定結果ではあまり大きな測定性能の影響がわからなかった.
その理由としては,ひとつは時間分解能の良さが指数関数のフィッティングではメリットとならなかったと考えられる.
それは指数関数にどのような関数を畳み込んでも指数関数となり,時間分解能自体が悪くても畳み込まれてしまうためである.
ただ,プラスチックシンチレータをハイレートで測定に使用できたため単純に統計量が違うにも関わらず性能があがっていない.
それは,フィッティングの定数項の影響などが考えられる.
が,きちんと説明しきれるものではなく完全な原因は不明である.

\subsection{g因子の測定}
今回のg因子の測定は理論と一致はしたが測定精度の限界よりDirac方程式の検証にとどまった.
今後はもう1桁測定精度を改善することにより,異常磁気能率の測定の領域にたどり着くことである.
この領域の測定をすることができればQEDの妥当性を検証できるようになる.
そのためには統計量を増やすだけでは充分でなく系統誤差を減らすことが大切となる.
特に今回の測定の系統誤差の要因となった,磁場の一様性の改善が大切と考えられる.
磁場標的の設計に際してより一様な磁場を設計すると共に,より高い精度で磁場を測定することが目標となる.

また,寿命と同様にg因子の測定は時間の測定の影響が大きい測定である.
そのため当然,同様の理由により測定結果はPS検出器の方がよくなると考えられる.
実際,寿命とは異なり$g$因子の測定はPS検出器の方が良い結果が得られている.
それはフィッティングの際に振動の項は三角関数によって表現されており,時間分解能の良さが直接に寄与することが要因と考えられる.
また,フィッティングのパラメータの数そのものも増えており,統計量の多さもメリットとなるためであると考えられる.

\subsection{ミッシェルパラメータの測定}
ミッシェルパラメータの測定値は大きな統計誤差,系統誤差共に持っているが,これは目標通りV-A理論の検証を行うことはできたと考えられる.
なぜならば,まず$\rho$の値が理論より0,0.75,1のいずれかに限定されており,この中から選択をするには充分な誤差であるためである.
一方,V-A理論のその大きな特徴であるパリティの破れの程度を表す$\xi$のみは,誤差が小さいためである.
これは$\xi$のみ,エネルギースペクトラムではなく計数比から計算されたためである.

今後の課題は当然より良い精度で測定することが考えられる.
よりよく測定することにより,異常磁気能率同様に弱い相互作用のより高次の項の影響を知ることができる.
ただし,これらの影響はミッシェルパラメータで表せる関数形ではなくなるため,このままの実験では難しいことが予想される.
特にV-A理論の先のワインバーグ=サラム理論の検証等を行うにはこのような実験では難しいと思われる.
ただし,単純に系統誤差を減らすという観点での目下の課題はNaI検出器の較正の手段についてである.
これは数十MeVの領域では線源等が使えず正確な較正手段がなく大きな系統誤差となってしまったである.
ここをよりよい較正手段を考案していけると良いと考えられる.

2つの検出器の比較でいうと,この測定ではエネルギーの測定が測定に大きな影響を与えるため当初NaI検出器のほうが良いと考えられていた.
それは,シンチレーション効率の良さから粒子数のゆらぎが小さくなるためである.
一方では,電磁シャワーの応答の観点ではPS検出器の方が優秀であるといえことが判明した.
それは原子番号の小さい物質のみで構成されるプラスチックの方が放出される制動放射線のエネルギーは低くなり,
主な電磁シャワーの漏れの原因である$\gamma$線が持ち去るエネルギーの絶対量が少なくなるためである.
そのため,PS検出器とNaI検出器の両方の結果の比較を行いたかった.
しかし,NaI検出器のみでしかきちんとした較正を行うことができなかった.
そのため,PS検出器の測定データが得られなかったのは残念である.

\subsection{まとめ}
全体としては実験を通じて主に弱い相互作用を中心とした標準模型の理論の理解をすすめることができたため良かったと思う.
また,ビームを用いた実験を行えたのは貴重な体験であったともに,実験に関して様々な経験をすることができたと思う.
大きく特性の異なる二種のシンチレータの測定器の設計や実際の測定データを通じて,それらの差異などの理解を深められたと思う.
