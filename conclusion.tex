\section{結論と考察}
全ての測定の解析の結果をまとめると表~\ref{tab:result_conclusion}にようになる.
ただし,プラスチックシンチレータについては解析手法A で得られた値を載せた.
いずれの測定結果においても,理論値と誤差の範囲で一致した結果が得られた.
ここでは,それぞれの結果の物理的な意味について確認し,今後の展望について考察していきたい.
また,2つの検出器の測定結果についての違いも考察する.

\begin{table}[h]
\centering
\caption{測定結果まとめ}
\label{tab:result_conclusion}
\begin{tabular}{ccccc}\toprule
{} & {} & 理論値($V-A$理論) & PS 検出器 & NaI 検出器\\ \midrule
\multicolumn{2}{c}{寿命~[ns]} &  2197 & $2222 \pm 6_{- 75}$ & $2184 \pm 52$\\
\multicolumn{2}{c}{$g$ 因子} & 2.002 & $2.004 \pm 0.003 \pm 0.017$ & $2.062 \pm 0.050^{+0.009}_{-0.020}$\\ 
{} & $\rho$ & 0.75 & --- & $0.66 \pm 0.02 \pm 0.14$\\
ミッシェルパラメータ & $\delta$ & 0.75 & --- & $0.64 \pm 0.12 \pm 0.20$\\
{} & $\xi$ & 1 & --- & $0.983 \pm 0.017$\\ \bottomrule
\end{tabular}
\end{table}%

\subsection{寿命の測定}
寿命の大きさを決定するのは主に相互作用の種類であり,結合が強ければ寿命は短くなる.
今回のミューオン崩壊は弱い相互作用によって起こるので,寿命は比較的長く,測定が可能だった.
実際の測定値も結合定数から予想される値と一致し,弱い相互作用の結合の弱さを検証できたと言える.
寿命の測定は基本的な定数である結合定数の測定として重要な手段である.
今回の測定では統計誤差が大きかったため,より統計を貯めることがより高精度の測定のために必要である.

寿命測定では2 つの観点からPS 検出器がNaI 検出器より良い測定結果をあたえることが予想された.
1 つは,PS はNaI に比べ信号の立ち上がり時間が短いため,NaI 検出器に比べ時間分解能が良いという点である.
もう1つは,PS はNaI に比べシンチレータ信号の立ち下がり時間が短いため,パイルアップの影響を受けづらく高レートの測定が可能であり統計量を増やしやすいという点である.
実際,寿命測定の統計誤差を見ると,PS のほうがNaI に比べて精度良く測定することができている.
その一方でPS ではフィッティング範囲に由来する系統誤差が大きくなっており,この点を改善していく必要がある.
今後の展望としては,PS と同様にNaI でも系統誤差のみが議論できる程度に統計を増やしていくとともに,解析手法を改良することで系統誤差も削減していくことがあげられる.%具体的にかけると◎

\subsection{$g$ 因子の測定}
今回の$g$ 因子の測定結果は,理論と一致はしたが測定精度の限界よりDirac 方程式の検証にとどまった.
今後の目標は,もう1 桁測定精度を改善することにより異常磁気能率の測定の領域にたどり着くことである.
この領域の測定をすることができればQED の妥当性を検証できるようになる.
そのためには統計量を増やすだけでは充分でなく系統誤差を減らすことが大切となる.
特に今回の測定の系統誤差の要因となった,磁場の一様性の改善が重要と考えられる.
磁場標的の設計に際してより一様な磁場を設計すると共に,より高い精度で磁場を測定することが目標となる.

また,寿命と同様に$g$ 因子の測定は時間の測定がより重要となる測定である.
そのため,同様の理由により測定結果はPS 検出器の方がよくなると考えられる.
実際,寿命と同様に$g$ 因子の測定はPS 検出器の方が良い結果が得られている.
また,フィッティングのパラメータ数そのものも増えており,統計量の多さがよりメリットとなったことも要因として挙げられる.

\subsection{ミッシェルパラメータの測定}
ミッシェルパラメータの測定値は大きな統計誤差,系統誤差を持っているが,目標通り$V-A$ 理論の検証を行うことはできたと考えられる.
まず$\rho$ の値が理論より0,0.75,1 のいずれかに限定されており,この中から選択をするには充分な誤差であるためである.
また,$V-A$ 理論の大きな特徴であるパリティの破れの程度を表す$\xi$ は,エネルギースペクトラムではなく計数比から計算したため誤差が小さくなっている.

より高い精度で測定することにより,異常磁気能率同様に弱い相互作用のより高次の項の影響を知ることができる可能性がある.
しかし,これらの影響はミッシェルパラメータで表せる関数形ではなくなるため,特に$V-A$ 理論の先のワインバーグ=サラム理論の検証等を行うにはこのような実験では難しいと思われる.
ただし,単純に系統誤差を減らすという観点ではNaI 検出器の較正の手段を改善することにより,それが可能である.
今回の実験では,数十~MeV の領域での正確な較正手段がなかったため,その点が大きな系統誤差となってしまった.
ここによりよい較正手段を考案することができれば系統誤差を削減することが可能である.

2つの検出器の比較でいうと,この測定ではエネルギーの測定が測定に大きな影響を与えるため当初NaI 検出器のほうが良いと考えられていた.
それは,シンチレーション効率の良さから光子数のゆらぎが小さくなるためである.
一方で,電磁シャワーの応答の観点ではPS 検出器の方が優秀であることが判明した.
これは原子番号の小さい物質のみで構成されるPS の方が,制動放射のエネルギーが低くなり,電磁シャワーの漏れの主な原因である$\gamma$ 線が持ち去るエネルギーの絶対量が少なくなるためである.
そのため,PS 検出器とNaI 検出器の両方の結果の比較を行いたかったが,NaI 検出器のみでしかきちんとした較正を行うことができなかった.
今後の目標はエネルギースペクトル測定の解析方法を考案し,PS 検出器でもきちんとエネルギーを測定することで,二つの検出器の結果を比較することである.

\subsection{まとめ}
実験を通じて,弱い相互作用を中心とした標準模型の理解をすすめることができた.
また,ビームを用いた実験を行えたのは貴重な体験であったとともに,実験に関して様々な経験をすることができた.
大きく特性の異なる二種のシンチレータの測定器の設計や実際の測定データを通じて,それらの差異などの理解を深めることができた.
