\documentclass[titlepage]{jsarticle}

\usepackage{here}
\usepackage{array,booktabs} %表のためのarray環境
%数式用パッケージ
\usepackage{amsmath,amssymb}
\usepackage{mathtools}
\usepackage{cancel}
\usepackage{cases}
\usepackage{bm}
\usepackage{colortbl}
\usepackage{float}
\usepackage{subcaption}
%ファインマングラフ用パッケージ
\usepackage{feynmf}
%ハイパーリンクの設定
\usepackage[dvipdfmx]{graphicx}
\usepackage[dvipdfmx]{hyperref}
\usepackage{pxjahyper}
\hypersetup{
  colorlinks=false, % リンクに色をつけない設定
  bookmarks=true, % 以下ブックマークに関する設定
  bookmarksnumbered=true,
  pdfborder={0 0 0},
  bookmarkstype=toc
}

\newcommand{\Slash}[1]{{\ooalign{\hfil/\hfil\crcr\(#1\)}}}
\newcommand{\figref}[1]{\figurename\ref{#1}}
\newcommand{\tabref}[1]{\tablename\ref{#1}}

%\graphicspath{{./figure/}}

\begin{document}
%title
\title{課題研究P2\\J-PARC MLF  のミューオンビームを用いた\\ミューオン崩壊に関する諸量の測定}
\author{阿部倫史,池満拓司,小田川高大,田島正規\\羽田野真友喜,早川龍,三野裕哉}
\date{\today}
\maketitle
%目次
\tableofcontents
\newpage
%
%\documentclass[titlepage]{jsarticle}

%\usepackage[dvipdfmx]{graphicx}

%数式用パッケージ
%\usepackage{amsmath, amssymb}
%\usepackage{mathtools}
%\usepackage{cancel}
%\usepackage{cases}
%\usepackage{bm}

%ファインマングラフ用パッケージ
%\usepackage{feynmf}

%ファインマンスラッシュ
%\newcommand{\Slash}[1]{{\ooalign{\hfil/\hfil\crcr\(#1\)}}}

%\title{課題研究P2\\J-PARC MLF  のミューオンビームを用いた\\ミューオン崩壊に関する諸量の測定}
%\author{阿部倫史,池満拓司,小田川高大,田島正規\\羽田野真友喜,早川龍,三野裕哉}
%\date{\today}

%\begin{document}

\section{序論}
本研究の目的はミューオンの崩壊現象を通して,標準模型,とくに弱い相互作用に関する各種の検証を行うことである.今回の実験で測定したい量は以下の三つである.

\begin{description}
\item[ミューオンの寿命]

ミューオンは弱い相互作用によって崩壊することが知られており,その寿命は標準模型によって計算されている.今回の実験ではミューオンビームを標的中で止めることで,ミューオンの寿命を測定し.その値を理論値と比較することで標準模型の検証を行う.
\item[ミッシェルパラメータ]

ミューオンの崩壊によって出てきた電子(陽電子)のエネルギースペクトルにおいて,ミッシェルパラメータと呼ばれるいくつかのパラメータを考えることができ,弱い相互作用の$V - A$ 理論ではこの値が決定されている.また,このエネルギースペクトルには$\cos \theta$ に比例する特徴的な項が存在し,これは弱い相互作用のパリティ対称性の破れを表している.今回の実験ではこのエネルギースペクトルを求めることでミッシェルパラメータの値を決定し,弱い相互作用のパリティ対称性の破れを確認するとともに,$V - A$ 理論の妥当性を検証する.
\item[$g$ 因子]

ミューオンの$g$ 因子の測定は標準模型の検証に用いられる典型的な実験の一つであり,とくに現在,標準模型を超えた物理の存在をも示唆する実験である.今回の実験ではミューオンビームを止める標的中に磁場をかけることでミューオンのスピンを歳差運動させ,$g$ 因子の測定をめざす.
\end{description}

また,これらの実験を通して粒子ビームを用いた素粒子実験に触れ,その方法を会得することを目的とする.

\section{理論}
本節では今回の実験で測定する物理量についてその背景にある理論を概説する.なお,詳しい計算等については一部については付録を見るか,あるいは参考文献を参照してほしい.以下,自然単位系($\hbar = c = 1$)を用いる.

\subsection{ミューオンとは}
ミューオン($\mu^{\pm}$)は1936年にAnderson らによって宇宙線から発見された,標準模型におけるレプトンの第二世代に属する素粒子である.$\mu^{\pm}$ は電子と同じ電荷$\pm e$,スピン$1/2$ を持ち,質量はその約$200$ 倍である.より正確には
	\[ m_{\mu} = 105.6583745 \pm 0.0000024~\mathrm{MeV}\]
と測定されている\cite{PDG}.
	
\subsection{ミューオンの寿命}
$\mu^{\pm}$ はほぼ100 \% の崩壊確率で次の崩壊を起こす\cite{PDG}.
\begin{eqnarray}
\mu^{-} \rightarrow e^{-} + \nu_{\mu} + \Bar{\nu}_{e}\\
\mu^{+} \rightarrow e^{+} + \Bar{\nu}_{\mu} + \nu_{\mu}
\label{eq:theory_muondecay}
\end{eqnarray}
今回の実験では$\mu^{+}$ を用いるので,以下$\mu^{+}$ についてその寿命を計算する.
	
\begin{figure}
\centering
\begin{fmffile}{feynmanzu1}
\begin{fmfgraph*}(120,80)

\fmfleft{mui}
\fmfright{numuo,eo,nueo}
				
\fmflabel{$\mu^{+}(p, r)$}{mui}
\fmflabel{$e^{+}(p', r')$}{eo}
\fmflabel{$\nu_e(q_{1}, r_{1})$}{nueo}
\fmflabel{$\bar{\nu}_{\mu}(q_{2}, r_{2})$}{numuo}
				
\fmf{fermion}{numuo,numumu,mui}
\fmf{boson,label=$W(k)$}{numumu,enue}
\fmf{fermion}{eo,enue}
\fmf{fermion}{enue,nueo}
				
\fmflabel{$\alpha$}{enue}
\fmflabel{$\!\!\!\beta$}{numumu}

\end{fmfgraph*}
\end{fmffile}
\vspace{10pt}
\caption{$\mu^{+}$ の崩壊のファインマン図}
\label{zu:muondecay}
\end{figure}
	
$\mu^{+}$ の崩壊は図\ref{zu:muondecay} のようなファインマン図で表される.標準模型(ワインバーグ=サラム理論)によればこの過程のファインマン振幅は
\begin{align}
\mathcal{M} = &-g_{W}^2\left[\Bar{u}(\bm{q_{1}})\gamma^{\alpha}(1 - \gamma_{5})v(\bm{p'})\right] \notag \\ 
&\times \frac{-(-g_{\alpha\beta} + k_{\alpha}k_{\beta}/m_{W}^2)}{k^{2} - m_{W}^2 + i\epsilon}\left[\Bar{v}(\bm{p})\gamma^{\beta}(1 - \gamma_5)v(\bm{q_{2}})\right]
\label{eq:theory_muondecayamp}
\end{align}%下の文章に対応する記号がありません
と書ける.ただし$p, q, k, \alpha, \beta$などは図\ref{zu:muondecay} に対応し,$g_{W}$ は弱い相互作用の結合定数である.その他の記法は付録に記載している通りである.ここで,$m_{W}^2$ が$k^2$ にくらべて十分大きいとし,$m_{W} \rightarrow \infty$の極限をとって計算すると,$V-A$理論の結果と一致する.この詳細な計算は付録に掲載するが結果として$\mu^+$ の寿命
\begin{equation}
\tau_{\mu} = \frac{192\pi^3}{G^{2} m_{\mu}^{5}}
\label{eq:thory_muonlifetime}
\end{equation}
が得られる.ここで$G$ はFermi 結合定数である.

実際の測定値としては
\[\tau_{\mu} = 2.1969811 \pm 0.0000022 \times 10^{-6}~\mathrm{s}\]
という値が得られている\cite{PDG}.
	
\subsection{ミッシェルパラメータ}
式\eqref{eq:theory_muondecay} で書かれる$\mu^{+}$ の崩壊で出てくる$e^{+}$ を考える.静止した$\mu^{+}$ が崩壊する時,運動量とエネルギーの保存から$e^{+}$ が持ちうる最大エネルギーは$m_{\mu}/2 \simeq 50~\mathrm{MeV}$ であり,完全に偏極された$\mu^{+}$ の崩壊を静止系で考えると,出てくる$e^{+}$ のエネルギー及び角度分布はミッシェルパラメータと呼ばれる四つのパラメータ$\rho, \eta, \xi, \delta$ を用いて次のようにあらわすことができる\cite{michel_parameter}.
\begin{align}
\frac{d^2\Gamma}{x^{2}dxd(\cos \theta)} \propto& \,\,(3 - 3x) + \frac{2}{3}\rho (4x - 3) \notag \\
&+ 3 \eta \,x_{0} \frac{1-x}{x} + \xi \cos \theta \left[(1 - x) + \frac{2}{3} \delta (4x - 3)\right]
\label{eq:theory_michel}
\end{align}
ここでニュートリノの質量や輻射補正は無視した.ただし,$\theta$ は$e^{+}$ の運動量と$\mu^{+}$ のスピンのなす角度であり,$x$ は$e^{+}$ のエネルギーの最大値が1 となるように規格化したものである.また,$x_0$ は$e^{+}$ のエネルギーが$m_{e}$ のときの$x$ の値であり,これは通常無視できるので第3 項は考えないことも多い.

$\theta = \pi/2$ の位置で,もしくは全$\theta$にわたって測定した場合にはスピンに関係する第4項が0となるため,第1 項,第2 項のみを考えればよい.このとき式\eqref{eq:theory_michel} から
\begin{equation}
\frac{d\Gamma}{x^{2}dx} \propto (3 - 3x) + \frac{2}{3}\rho (4x - 3)
\label{eq:theory_michel2}
\end{equation}
となる.

Lorentz 共変な相互作用においては$\rho = 0, 0.75, 1$ のいずれかになることが分かっている\cite{michel_interaction}.入射粒子数で規格化すると,結局$\Gamma$ は$\rho$ によらないので式\eqref{eq:theory_michel2} をそのまま描いて,測定で得られるグラフの形は図\ref{zu:michelpar} のいずれかのようになる.
\begin{figure}[htbp]
\centering
\includegraphics[width = 0.5\textwidth]{figure/abe/michelgraph.png}
\caption{Michel 崩壊の各$\rho$ に対するスペクトル}
\label{zu:michelpar}
\end{figure}
これらのグラフは大きく形が異なっているため,測定で得られたスペクトルを見ればある程度の相互作用の形を知ることができる.

特に,$V-A$ 理論では
\[ \rho = \xi\delta = \frac{3}{4},\, \eta = 0,\, \xi = 1 \]
となる.

また,実際には輻射補正を考えるとたとえば$\rho$ は$3 \sim 7$~\% ほど小さく測定されることが知られている.

現在のミッシェルパラメータの測定値は
\begin{align*}
\rho &= 0.74979 \pm 0.00026\\
\eta &= 0.057 \pm 0.034\\
\delta &= 0.75047 \pm 0.00034\\
\xi &= 1.0009^{+0.0016}_{-0.0007}
\end{align*}
である\cite{PDG}.
	
\subsection{ミューオンの$g$因子}
$\mu^{+}$ を含む,ディラック場で表されるようなフェルミ粒子の運動は次のDirac 方程式で記述される.
\begin{equation}
(i\Slash{\partial} - m)\psi(x) = 0
\label{eq:theory_dirac}
\end{equation}
ここで記法は付録と同じとする.

次に式\eqref{eq:theory_dirac} をもとにして外場としての電磁場$A_{\mu}$ が存在する状態での方程式を考える.U(1) 局所ゲージ対称性を課すと,共変微分$D_{\mu} = \partial_{\mu} + ieA_{\mu}$ を定義できて,
\begin{equation}
(i\Slash{D} - m)\psi(x) = 0
\label{eq:theory_diracwitha}
\end{equation}
となる.

ここから非相対論的近似を行うために,いままで用いてきた自然単位系$\hbar = c = 1$ から$\hbar$ と$c$ を復活させる.$A_{\mu} = (\phi, \bm{A})$ とおくと,
\begin{equation}
i\hbar\frac{\partial}{\partial t}\psi(x) = \left\{c\bm{\alpha}\cdot (\bm{p} - e\bm{A}) + \beta m c^{2} + e \phi\right\}\psi(x)
\label{eq:theory_diracwithaphi}
\end{equation}
となる.スピノル$\psi(x)$ を
\begin{equation}
\psi(x) = \begin{pmatrix}
\chi_{1} (x)\\
\chi_{2} (x)
\end{pmatrix} \exp \left( - i \frac{mc^{2}}{\hbar} t\right)
\end{equation}
と書き,式\eqref{eq:theory_diracwithaphi} に代入すると
\begin{align}
i\hbar\left( -i\frac{mc^{2}}{\hbar} + \frac{\partial}{\partial t}\right) \chi_{1} = c\bm{\sigma}\cdot(\bm{p} - e \bm{A})\chi_{2} + (mc^{2} + e\phi)\chi_{1} \label{eq:theory_chi1}\\ 
i\hbar\left( -i\frac{mc^{2}}{\hbar} + \frac{\partial}{\partial t}\right) \chi_{2} = c\bm{\sigma}\cdot(\bm{p} - e \bm{A})\chi_{1} - (mc^{2} - e\phi)\chi_{2} \label{eq:theory_chi2}
\end{align}

非相対論近似においては式\eqref{eq:theory_chi2} について時間微分項が無視でき,またスカラーポテンシャルによるエネルギーも無視できるので
\begin{equation}
\chi_{2} = \frac{1}{2mc}\bm{\sigma}\cdot(\bm{p} - e\bm{A})\chi_{1}
\end{equation}
となる.これを式\eqref{eq:theory_chi1} に代入し,$\sigma$ 行列に関する計算を行えば
\begin{equation}
i\hbar\frac{\partial}{\partial t}\chi_{1} = \left\{\frac{(\bm{p} - e\bm{A})^{2}}{2m} + e\phi - \frac{e\hbar\bm{\sigma}}{2m}\cdot\bm{B}\right\}\chi_{1}
\label{eq:theory_pauli}
\end{equation}
という,Pauli 方程式が得られる.ここで$\bm{B} = \nabla \times \bm{A}$ であり,磁束密度を表す.式\eqref{eq:theory_pauli} よりスピン$\hbar\bm{\sigma}/2$ がもつ磁気モーメントは軌道角運動量の場合の二倍であり,この場合$g$ 因子は2 であることがわかる.

上のような計算を行えばDirac 方程式からミューオンの$g$ 因子は2 と求まる.QED (量子電磁力学)による考察を行うと,その結果$g$ 因子には補正がかかることが分かっている.$g$ 因子のこの2 からのずれを異常磁気能率 (anomalous magnetic moment) と呼ぶ.例えば二次の摂動において異常磁気能率に寄与する過程はSchwinger によって説明された図\ref{zu:vertexcorr} で表される過程である.

\begin{figure}[h]
\centering
\begin{fmffile}{feynmanzu2}
\begin{fmfgraph*}(100,80)
				
\fmfleft{mui}
\fmfright{muo}
\fmfbottom{ai}
				
\fmflabel{$\mu$}{mui}
\fmflabel{$\mu$}{muo}
\fmflabel{$\gamma$}{ai}
				
\fmf{fermion}{mui,muig,muia}
\fmf{boson}{ai,muia}
\fmf{boson,left=0.5,tension=0.2}{muig,muog}
\fmf{fermion}{muia,muog,muo}
				
\end{fmfgraph*}
\end{fmffile}
\vspace{10pt}
\caption{二次の摂動における異常磁気能率への寄与過程}
\label{zu:vertexcorr}
\end{figure}

図\ref{zu:vertexcorr} で表される過程の寄与は$\alpha/2\pi$ (ここで$\alpha$ は微細構造定数$1/137$)である.

このような寄与をQED のより高次の摂動についても考えることができ,また一方でQED のレプトニックな過程以外の$W, Z$ ボソンを含む電弱理論に関して,あるいはハドロンが関与する過程に関しても考えることができる.このようにして,さまざまな相互作用を加味した異常磁気能率の理論値\cite{g-2_theory}は
\[a_{\mu} = (g -2)/2 = 116591804(51) \times 10^{-11}\]
であり,その測定値\cite{g-2_experiment}は
\[a_{\mu} = 116592089(63) \times 10^{-11}\]
である.%\cite{}
これらはQED の正しさを証明する一方で,標準模型の理論値とも$3\sigma$ 以上の有意な差があり,ここに新たな物理があることが期待されている.実際,未発見の粒子が存在して上に述べたような過程の他にその粒子の寄与を考えると,計算するべきファインマン図が増え,その分異常磁気能率に対する計算値も変わることになる.

そのため,$g$ 因子の測定は新しい物理の探索を目的として現在の素粒子物理学実験においてもっとも精密な計算と測定が行われている例である.

\section{実験原理}
本節では今回の実験の原理について説明する.今回の実験ではミューオン ($\mu^{+}$) ビームを用いてミューオンの崩壊寿命,ミッシェルパラメータ,そして$g$ 因子を測定する.そのために用いる検出器としてNaI (Tl) シンチレータとプラスチックシンチレータを採用する.なお,ミューオンビームの詳細は次節に含めている.

\subsection{寿命測定の原理}
ミューオンビームは標的で止められる.静止したミューオンは式\eqref{eq:thory_muonlifetime}の寿命を持つため,
\begin{equation}
\frac{dN_\mu}{dt} = -\frac{1}{\tau_\mu} N_{\mu}
\end{equation}
に従い崩壊し,この際にミューオンはほぼ100~\%の確率で陽電子を一つ出す.崩壊して出てくる陽電子の数は崩壊したミューオンの数と一致し
\begin{equation}
\frac{dN}{dt} = N_0 \exp{\left(-\frac{t}{\tau_\mu}\right) }
\end{equation}
で減少する.つまり陽電子を検出し,計数の時間変化を指数関数でフィッティングすれば寿命を求めることができる.

\subsection{ミッシェルパラメータ測定の原理}
今回はミッシェルパラメータのうち$\rho$ を測定するための実験を行なった.理論の節と述べたように$\rho$ の測定にはミュオンのスピンの向きに対して,$\theta = \pi / 2$ の位置で観測,または無偏極のミューオンのエネルギーを測定しなくてはならない.

ミュオンビームは進行方向にスピン偏極しているため,標的からビーム方向に対して$90^{\circ}$ の向きに検出器を設置してエネルギーを測定,得られたエネルギースペクトルを式\eqref{eq:theory_michel2} でフィッティングすれば,$\rho$ を求めることができる.この際,最大50~MeVの陽電子が出るので,50~MeV陽電子を止めきるような検出器が要求される.また実際には,$90^{\circ}$ 方向に検出器を置くことが困難だったことやデータ量の関係から,後述する$g$ 因子測定のデータを利用して,全スピン方向で積分した無偏極ミューオンとして$\rho$ を求めた.また,この解析によりスピン部分のデータが得られたため,最終的にはミッシェルパラメータの$\xi,\;\delta$ の解析も行った.

\subsection{$g$ 因子測定の原理}
ミュオンのスピンは磁場中で歳差運動をする.一様磁場中において,磁場方向を$z$軸とすると,磁場とスピンの相互作用のハミルトニアンは,
\begin{equation}
\hat{H} = - g \frac{e}{2m_\mu} \hat {\textsl{\textbf {S}}} \cdot \textsl{\textbf {B}} = - g \frac{e}{2m_\mu}  \hat{S_z} B
\end{equation} 
であるので,Heisenberg方程式より,
\begin{equation}
\frac{d\hat{S_x}}{dt} =  g \frac{eB}{2m_\mu}\hat{S_y} 
\end{equation} 
\begin{equation}
\frac{d\hat{S_y}}{dt} = - g \frac{eB}{2m_\mu}\hat{S_x} 
\end{equation} 
\begin{equation}
\frac{d\hat{S_z}}{dt} = 0 
\end{equation} 
を得る.スピン初期状態を
\begin{equation}
\langle \textsl{\textbf {S}}(t=0)\rangle = (C,0,0)
\end{equation} 
とすれば,
\begin{equation}
\langle S_x \rangle= C\cos{(g\frac{eB}{2m_\mu}t)}
\end{equation}
\begin{equation}
\langle S_y \rangle= -C\sin{(g\frac{eB}{2m_\mu}t)}
\end{equation}
とスピンが$xy$平面内で回転することがわかり,その角速度$\omega$は,
\begin{equation}
\omega = g\frac{eB}{2m_\mu}
\end{equation}
となる.式\eqref{eq:theory_michel} の通り陽電子はミューオンのスピンの方向に出やすいので,磁場を通した標的でミューオンビームを止めると,陽電子の計数は指数的な減少に周期$2\pi / \omega$ の振動が加わったものになる.よってその周期から$g$ 因子を求めることができる.

\subsection{検出器サイズの見積もり}
今回の実験ではミューオンの崩壊の際に放出される陽電子を測定するが,この陽電子の最大エネルギーはおよそ50~MeVである事を前節で確認した.
もし検出器に50~MeVの陽電子が物質に入射すると,制動放射と対生成による電磁シャワーを形成する.今回の検出器は全吸収型のカロリメータとして設計したため,その電磁シャワーが検出器の寸法内に収まるように設計しなければならない.電磁シャワーの広がりはその内部にシャワーのエネルギーの90~%が含まれるような長さで記述され,その長さをモリエール半径$R_\mathrm{M}$ と呼ぶ.モリエール半径は次のように定義される.
\begin{equation}
R_\mathrm{M} = L_\mathrm{rad}\frac{21.2~\mathrm{MeV}}{E_\mathrm{c}}
\end{equation}
ここで$L_\mathrm{rad}$ は放射長,$E_\mathrm{c}$ はcritical energy である.NaI およびプラスチックシンチレータ (PS) における$L_{\rm rad}$ と$E_\mathrm{c}$ ,$R_\mathrm{M}$ の値を表\ref{tab:abe_rm} に示す.
\begin{table}[hbtp]
\centering
\caption{物質ごとの$L_\mathrm{rad}$と$E_\mathrm{c}$,およびモリエール半径$R_\mathrm{M}$}
\begin{tabular}{cccc}\toprule
物質 & $L_\mathrm{rad}~[\mathrm{cm}]$ & $E_\mathrm{c}~[\mathrm{MeV}]$ & $R_\mathrm{M}~[\mathrm{cm}]$ \\ \midrule
NaI & 2.59 & 17.4 & 3.2 \\
PS & 42.9 & 109 & 8.34 \\ \bottomrule
\end{tabular}
\label{tab:abe_rm}
\end{table}
この値から,検出器の横幅はNaIは5~cm,プラスチックシンチレータは10~cm程度の半径でよいと見積もった.なお実際にはGeant4 のシミュレーション結果も用いて,奥行き方向の長さも含めた検出器サイズを決定した.
% 奥行き方向に関する理論はなかったということ?
%ーLeoに乗っている奥行き方向に関する簡単なモデルではよい計算結果は出ないです,Leoで次に書いてあるのもシミュレーションなのでシャワーの奥行きについての理論的見積もりはないとしました

%\end{document}

%\documentclass[]{jsarticle}

%\usepackage[dvipdfmx]{graphicx}
%\usepackage[dvipdfmx]{color}

%\usepackage{amsmath, amssymb}
%\usepackage{mathtools}
%\usepackage{cancel}
%\usepackage{cases}
%\usepackage{bm}

%\usepackage{here}
%\usepackage{colortbl}

%\begin{document}

%----------ここから-------------------
%----------ミューオンビーム-------------------
\section{実験方法}
\subsection{MLFミューオンビーム}
\subsubsection{加速器科学インターンシップの利用}
KEKが学部3回生以上を対象に行っている加速器科学インターンシップを利用することにより,ロシアの実験チームの MLF 実験課題 2017B0163 のパラサイト実験という形でMLFミューオンビームを利用できることを知った.ミューオンビームの性能を踏まえて可能な測定量および測定方法を考え実験の準備を行い、そのインターンシップを用いて実際のMLFミューオンビームを用いて測定を行った.
 \subsubsection{表面ミューオン}
 MLFでは炭素原子核に高エネルギーの陽子を衝突させることによってパイオンを生成し,パイオンが崩壊して得られるミューオンを利用している.炭素標的から飛び出したパイオンが超伝導ソレノイド磁石内部で崩壊することによって得られるミューオンは崩壊ミューオンと呼ばれるが,今回利用したのは炭素標的の表面に静止した $\pi^+$ 中間子の崩壊によって得られる$\mu ^+$で,これは表面ミューオンと呼ばれる.この表面ミューオンは静止したパイオンから生じているため 100\% のスピン偏極を持っており,非常にエネルギーが低く一定であるという特徴を有する.表面ミューオンがスピン偏極を持つのは弱い相互作用による崩壊で生じる際に、ニュートリノはヘリシティーが左巻きのものが結合することに由来する.なお炭素標的の表面で静止した $\mu^-$ 中間子は原子核に捕獲されるため,取り出すことはできない.\par
 表面ミューオンビームラインの性能は表\ref{muon1}のとおりで,シングルバンチ(短時間のミューオンの集まり)のミューオンビームが $25 (\mathrm{Hz})$でやってくる.ビームの広がりのプロファイルは図\ref{muon2}のとおりである.%要表記確認


\begin{figure}[H]
  \centering
  \includegraphics[width=0.4\textwidth]{figure/hayakawa/decay_pion.png}
  \caption{$\pi^+$ 中間子の崩壊}
\end{figure}

  \footnotetext{上図は http:\slash\slash{}slowmuon.kek.jp\slash{}aboutMuon.htmlより引用 } % 画像引用は参考文献?
  

      
  \begin{table}[H]
    \caption{表面ミューオンビームラインの性能}
    \label{muon1}
    \centering
    \begin{tabular}{|c|c|}\hline
      ビームエネルギー & 4.1 (MeV) \\ \hline
      侵入長 & $\sim$ 0.2 (mm) \\ \hline
      エネルギー分布 & $\sim$ 15  \% \\ \hline
      パルス幅 (FWHM) & $\sim$ 100 (ns) \\ \hline
      ビームサイズ & 30 (mm) $\times$ 40 (mm) \\ \hline
      ビーム強度 & 3 $\times$ $10^7$ (/s) \\ \hline
      ポート数 & 2 \\ \hline
    \end{tabular}
  \end{table}
   
  \begin{figure}[H]
    \centering
    \includegraphics[width=0.8\textwidth]{figure/hayakawa/profile.pdf}
    \caption{ミューオンビームの広がり(単位:$\mathrm{mm}$)}
    \label{muon2}
  \end{figure}
  
  %---------実験の方法-------------------
  \subsection{測定量と検出器}
  \subsubsection{実験概要}

今回の実験ではミューオンの寿命,ミッシェルパラメータ,$g$因子を測定したい.基本的な実験の流れとしては,
       \begin{itemize}
        \item Beam Line から $\mu ^+$ が出て来る
        \item ターゲットに止められた $\mu ^+$ が e$^+$に崩壊する
        \item 検出器で時間情報・エネルギー情報を測定する
        \end{itemize}
という順序になる.測定の時間情報は寿命とg$因子$の測定に,エネルギー情報はミッシェルパラメータの測定にと各解析に対して独立に必要である.そのためにそれぞれの測定を中心に行う検出器として,時間分解能に優れたプラスチックシンチレータ(PS)検出器および,エネルギー分解能に優れたNaIシンチレータ検出器の二種類の検出器を作成した.
       
    \begin{figure}[H]
      \centering
      \includegraphics[width=1\textwidth]{figure/hayakawa/lifetime.png}
      \caption{実験概要}
  \end{figure}

  \subsubsection{検出器サイズの見積}

図 \ref{PS_sim} はPS検出器の体積シミュレーションである.検出器サイズの縦横は $20 (\mathrm{cm})$ で固定し,奥行きを $20 (\mathrm{cm})$ から $24 (\mathrm{cm})$ まで変化させた直方体状のPSシンチレータに測定すべき最大のエネルギーである$50 (\mathrm{MeV})$ の陽電子を入射させた時に検出器に落とすエネルギーをシミュレーションした結果をヒストグラムで示している.奥行きが$24 (\mathrm{cm})$以下では$50 (\mathrm{MeV})$より下にピークが存在し,電磁シャワーが寸法内に収まらず陽電子のエネルギー充分に検出器に落とせていないことが分かる.一方,奥行きを $24 (\mathrm{cm})$ 以上に増やして,漏れるエネルギーが光子によるものの影響のためほとんど落とすエネルギーが変わらない.つまり,これ以上大きくしても効率が悪く,また基本的には$50 (\mathrm{MeV})$程度エネルギーを落としているため,奥行きは$24 (\mathrm{cm})$ で決定した.

  \begin{figure}[H]
    \centering
    \includegraphics[width=0.6\textwidth,angle=-90]{figure/hayakawa/pl_20_24.pdf}
    \caption{PS検出器の体積シミュレーション}
    \label{PS_sim}
  \end{figure}

図 \ref{NaI_sim} はNaI検出器の体積シミュレーションである.NaI は光電子増倍管 (PMT) の接続された既製品を利用したため,既製品をどのように並べるべきか確認するために縦横の幅を変えながら同様のシミュレーションを行った.

  \begin{figure}[H]
    \centering
    \includegraphics[width=0.55\textwidth,angle=-90]{figure/hayakawa/NaI_10_20.pdf}
    \caption{NaI検出器の体積シミュレーション}
    \label{NaI_sim}
  \end{figure}

  %---------検出器の製作-------------------
\subsection{検出器の製作}
\subsubsection{PS検出器の製作}

光ファイバー読み出しの板を並べることで,縦横 $20 (\mathrm{cm})$ 奥行き $24 (\mathrm{cm})$ の体積のPS検出器を作成した.

  \begin{figure}[H]
        \centering
        \includegraphics[width=0.3\textwidth]{figure/hayakawa/psmd.png}
        \caption{PS板}
        \centering
        \includegraphics[width=0.8\textwidth]{figure/hayakawa/p7.png}
        \caption{PS検出器寸法}
        \label{PS_sunpou}
  \end{figure}

以下の順序で検出器を作成した.
  \begin{itemize}
    \item 厚み$6(\mathrm{cm})$ に束ねたものを$4$セット作成した
    \item 光ファイバーの片端をクッキーを用いて光学セメントで固定した
       \begin{figure}[H]
         \centering
         \includegraphics[width=0.5\textwidth]{figure/hayakawa/ps_kotei.jpg}
         \caption{光ファイバーの固定の様子}
       \end{figure}
    \item PSに光ファイバーを貫通させ,もう片端も固定した
    \item 光量を最大限確保するため,クッキーの端面を研磨した
    \item クッキーの端面と PMT の境界に光学グリスを塗り接続した
    \item 暗箱内に設置するための枠に収めた.枠とPMTの結合にはアルミU型チャンネルを用いた
       \begin{figure}[H]
         \centering
         \includegraphics[width=0.6\textwidth]{figure/hayakawa/waku1.png}
         \caption{暗箱に固定するための枠の設計}
       \end{figure}
    \item 暗箱内に設置し前方にコリメータを配置した
  \end{itemize}



\begin{figure} [H]
    \centering
    \includegraphics[width=0.6\textwidth]{figure/hayakawa/PS_real.jpg}
    \caption{PS検出器外観}
\end{figure}

\begin{figure}[H]
  \centering
  \includegraphics[width=0.7\textwidth]{figure/hayakawa/PS_in.jpg}
  \caption{PS検出器内部}
\end{figure}




\subsubsection{NaI検出器の製作}
$5.6(\mathrm{cm})\times 5.6(\mathrm{cm})\times 15(\mathrm{cm})$のNaI(Tl)の結晶がPMTに接続されたもの(以下,NaIとよぶ)を$3\times 3$個並べ,検出器の前面中央の前には$4(\mathrm{cm})\times 4(\mathrm{cm})$のプラスチックシンチレータで作ったトリガー用カウンターを設置した.

 \begin{figure}
    \begin{tabular}{cc}
      \begin{minipage}{0.5\hsize}
        \centering
        \includegraphics[width=0.8\textwidth]{figure/hayakawa/p6.png}
        \caption{NaI寸法(mm)}
        \end{minipage}
        \begin{minipage}{0.5\hsize}
        \centering
        \includegraphics[width=0.8\textwidth]{figure/hayakawa/NaI_real.jpg}
        \caption{NaI外観}
      \end{minipage}
    \end{tabular}
  \end{figure}

\subsection{架台の製作}
ビームの出る高さが地表から $1565 (\mathrm{mm})$ なので,検出器を置くためにはその高さに対応した架台が必要であった.そこでアルミフレームの一種であるレコフレームを使用して架台を作成した.アジャスタ付きキャスタを用いることによって高さは微調整可能なように設計した.現場ではレーザーを用いて水平および垂直方向の位置調整を行った.


  \begin{figure}[H]
    \begin{tabular}{cc}
      \begin{minipage}{0.5\hsize}
        \centering
        \includegraphics[width=0.9\textwidth]{figure/hayakawa/kadai_setup.jpg}
        \caption{架台の組み立て}
        \end{minipage}
        \begin{minipage}{0.5\hsize}
        \centering
        \includegraphics[width=0.9\textwidth]{figure/hayakawa/laser1.jpg}
        \caption{レーザーを用いた位置調整}
      \end{minipage}
    \end{tabular}
  \end{figure}
  
      \begin{table}[H]
      \caption{3種類の架台の外寸}
      \label{tab:kadai}
      \centering
      \begin{tabular}{|c|c|}\hline
        &  W $\times$ D $\times$ H ($\mathrm{mm}$)\\ \hline
        PS検出器架台 &  1200 $\times$ 600 $\times$ 1283\\ \hline
        NaI検出器架台 & 600 $\times$ 600 $\times$ 1358 \\ \hline
        ターゲット架台 & 600 $\times$ 600 $\times$ 1331 \\ \hline
      \end{tabular}
    \end{table}

\subsection{Waveform Digitizer}
データ測定については波形をそのまま記録することができるWaveform Digitizer(以下,WFDとよぶ) をPS検出器およびNaI検出器用にそれぞれ利用した.測定開始の外部トリガには,加速器ライン側のビームの発射信号を遅延させて陽電子の崩壊の検出のスタートよりわずか手前になるように調整して入力している.
\subsubsection{WFD(CAEN Waveform Digitizer V1721)}
  \begin{itemize}
    \item 8channel 8bit 500(MS/s) Digitizer
    \item 時間分解能が良いので,主に崩壊寿命測定用のPSの信号に用いた
  \end{itemize}
  \begin{figure}[H]
    \begin{tabular}{cc}
      \begin{minipage}{0.5\hsize}
        \centering
        \includegraphics[width=0.8\textwidth,angle=-90]{figure/hayakawa/ps_plot.pdf}
        \caption{PS用のWFDで記録した波形}
      \end{minipage}
      \begin{minipage}{0.4\hsize}
        \centering
        \includegraphics[width=0.2\textwidth]{figure/hayakawa/1095_L.jpg}
      \end{minipage}
    \end{tabular}
  \end{figure}




\subsubsection{WFD(CAEN Waveform Digitizer DT5725)}
  \begin{itemize}
    \item 8channel 14bit 250(MS/s) Digitizer
    \item エネルギー分解能が良いので,主にエネルギー測定用のNaI検出器の信号の記録に用いた
    \item 9本のNaIに対して8chなので,アナログ信号を合成して入力した
  \end{itemize}
  \begin{figure}[H]
    \begin{tabular}{cc}
      \begin{minipage}{0.5\hsize}
        \centering
        \includegraphics[width=0.8\textwidth,angle=-90]{figure/hayakawa/NaI_plot.pdf}
        \caption{NaI用のWFDで記録した波形}
      \end{minipage}
      \begin{minipage}{0.4\hsize}
        \centering
        \includegraphics[width=0.6\textwidth]{figure/hayakawa/DT5725_L.png}
      \end{minipage}
    \end{tabular}
  \end{figure}




%----------ここまで-------------------
%\end{document}

%\documentclass[uplatex,dvipdfmx]{jsarticle}
 
%\usepackage[dvipdfmx]{graphicx}
%\usepackage{graphicx}
%\usepackage{mathtools}
%\usepackage{cancel}
%\usepackage{amsmath,amssymb}
%\usepackage{cases}
%\usepackage{bm}

%\usepackage{here}
%\usepackage{colortbl}
%\usepackage{feynmf}

%\newcommand{\Slash}[1]{{\ooalign{\hfil/\hfil\crcr\(#1\)}}}

%\begin{document}

\newpage
%\section{実験}

\subsection{寿命測定のための銅板標的}

寿命測定に用いた銅板標的を図~\ref{tar_cu} に示す.
これは厚さ$0.6~\mathrm{mm} \times$ 横$280~\mathrm{mm} \times$ 縦$120~\mathrm{mm}$ の銅板を木枠に固定したものである.銅板の縦横の長さはビームプロファイルから計算されるビームの広がりに対して約$3\sigma$ になるように決めた.また厚さは$4~\mathrm{MeV}$ ミューオンが銅板厚み方向の中心付近で止まるようなものを選んだ.

\subsection{$g$ 因子測定のための磁場装置}

$g$ 因子測定に用いた磁場印加標的を図~\ref{tar_mag} に示す.
厚さ$0.6~\mathrm{mm} \times$ 横$80~\mathrm{mm} \times$ 縦$60~\mathrm{mm}$ の銅板を,呼び径$200~\mathrm{mm}$ の塩化ビニルパイプの中心にくるように紐で吊るした.銅板の縦横の大きさはミューオンビームの広がりに対して約$1\sigma$ の大きさになるように決めた.パイプ内側に計36 個の永久磁石を接着剤で貼り付け,テープで補強した(磁石の配置については後述).永久磁石にはセラミック磁石 (Y25) を用いており,磁石一個あたりの大きさは厚さ$9~\mathrm{mm} \times$ 横幅$10~\mathrm{mm} \times$ 長さ$60~\mathrm{mm}$ である.

\begin{figure}[H]
\begin{minipage}{0.45\hsize}
\begin{center}
\includegraphics[width=1\textwidth]{figure/tajima/cu_target.jpg}
\caption{銅板標的.}
\label{tar_cu}
\end{center}
\end{minipage}
\hfill
\begin{minipage}{0.45\hsize}
\begin{center}
\includegraphics[width=1\textwidth]{figure/tajima/mag.jpg}
\caption{磁場発生装置と標的 \protect\footnotemark.}
\label{tar_mag}
\end{center}
\end{minipage}
\end{figure}
\footnotetext{写真では下方の磁石が外れているが,これは実験前に修復した.}

\subsubsection{磁場の発生原理について: $\cos \theta$配置}
磁場の生成には$\cos n \theta$ 巻き ($n=1$) の電磁石を参考にした.$\cos n \theta$ 巻きの電磁石の考え方は以下のものである.図~\ref{cos2}, \ref{cos4} のように,十分長い円環上を電流が$z$ 軸方向に流れているとする.この電流密度が$\cos n\theta$ (ここで$\theta$ は円柱座標 ($r,\theta,z$) における$\theta$ のことで,$n$ は整数)に比例したとする.すると原点付近の領域に, $2n$ 極磁場が形成されるというものである.導出の概略は,円環各点が形成する磁場をその点の周りでテーラー展開し,これを$\theta$ について積分すると$2n$ 極磁場を形成する項のみが残るというものである\cite{magnet}.
\begin{figure}[H]
\begin{minipage}{0.45\hsize}
\centering
\includegraphics[width=0.8\textwidth]{figure/tajima/cos.png}
\caption{$\cos\theta$ : 2 極磁石\cite{magnet}.}
\label{cos2}
\end{minipage}
\hfill
\begin{minipage}{0.45\hsize}
\centering
\includegraphics[width=0.7\textwidth]{figure/tajima/cos2.png}
\caption{$\cos 2 \theta$ : 4 極磁石\cite{magnet}.}
\label{cos4}
\end{minipage}
\end{figure}

作成の手軽さから,磁場発生源には電磁石ではなく永久磁石(セラミック磁石)を用いた.電磁石における電流密度を,永久磁石の分布密度で置き換え,磁石の分布密度が$|\sin\theta|$ に比例するようにした.$\sin\theta$ としたのは磁石の向きをすべて動径方向に向けて配置すると,電磁石の場合に比べて磁場の向きが$\pi/2$ だけずれるためである.また各磁石の磁場方向は$0 < \theta < \pi$ と$\pi < \theta < 2\pi$ で動径方向に対して正と負になるようにした.

作成に入る前に,考えた配置をFEMM を用いてシミュレートし,磁場の一様性を確認した.図~\ref{femm}はFEMM によるシミュレーション結果で、赤線の太枠で囲われた領域が銅板に対応している。実際に有限の長さの磁石で磁場装置を作成するにあたって,図~\ref{tar_mag} のように磁石を長手方向に$3~\mathrm{cm}$ の間隔を開けて配置した.これは,間隔を開けることにより長手方向に隙間を空けずに磁石を詰めたときに比べて,磁力線が緩和され長手方向に磁場の一様性が増すと考えたためである.

\begin{figure}[H]
\centering
\includegraphics[width=0.9\textwidth]{figure/tajima/femm_modify1-1.pdf}
\caption{FEMMによる2次元シミュレーション結果.真ん中の赤線で囲まれた部分が銅板標的に対応する.}\label{femm}
\end{figure}


\newpage

\subsection{予備実験}

\subsubsection{プラスチックシンチレータの宇宙線較正}
プラスチックシンチレータでミューオンの崩壊から生じる陽電子のエネルギーを測定するために, 宇宙線を用いてエネルギー較正を行った.

%%%%%%挿入&修正by池満%%%%%%%%
ケーブル類とHV 値は本実験の状態と同様の状態にしたうえで,本実験において陽電子が入射する面を上にむけて宇宙線の測定を行った.各PMT にかけた電圧を表~\ref{PS_PMT_HV} に示す.データ取得の条件は,WFD のチャンネルトリガーを用いて全チャンネルのORでトリガーをとり,1 波形分を取れるような時間幅としてゲートは208~ns とした.このときのトリガーのしきい値は全チャンネルで共通にしており,電圧値は各チャンネルのレートが約$3 \sim 5~\mathrm{Hz}$になるように選んだ.
\begin{table}[H]
\caption{プラスチックシンチレータ用のPMT のHV 値}
\label{PS_PMT_HV}
\centering
\begin{tabular}{cc}\toprule
PMT 名 & HV値 \\ \midrule
チャンネル0(あけみ)& $-1600~\mathrm{V}$ \\
チャンネル1(勝太郎)& $-1418~\mathrm{V}$ \\
チャンネル2(畑さん)& $-1940~\mathrm{V}$ \\
チャンネル3(紗智子)& $-1762~\mathrm{V}$ \\
チャンネル4(蘭)  & $-1871~\mathrm{V}$ \\
チャンネル5(矢部) & $-1947~\mathrm{V}$ \\
チャンネル6(政子) & $-1980~\mathrm{V}$ \\
チャンネル7(王)  & $-1905~\mathrm{V}$ \\ \bottomrule
\end{tabular} 
 \end{table}%

得られたWFD の電圧信号を時間積分し,さらに抵抗値$50~\Omega$ で割ることで,
信号の電荷量~[pC] を得た.各チャンネル毎の電荷量のヒストグラムは図~\ref{cosmicray} のようになり,図中の3 つのピークの由来を左からそれぞれペデスタル,環境放射線,宇宙線と考えた.
\begin{figure}[H]
\centering
\includegraphics[height = 0.9\columnwidth , angle = -90]{figure/ikemitsu/cosmicray_modify.pdf}
\caption{宇宙線測定で得られた各チャンネルの電荷量のヒストグラム.横軸は信号の電荷量の積分値~[pC],縦軸はカウント数である. }
\label{cosmicray}
\end{figure}%

宇宙線によるピークの部分を,Landau 関数をガウシアンで畳み込み積分した関数(以下,Langauss 関数と呼ぶ.)でフィッティングを行った.ガウシアンで畳み込み積分をしたのは,プラスチックシンチレータとPMT で測定されるエネルギーには比較的大きいゆらぎがあるからである.そしてフィッテイングにより得られた最頻値をエネルギー損失12~MeV に対応する電荷量とした.ただし,最頻値に対応するエネルギー損失の見積もりは次の2つの仮定を基にしている.
\begin{itemize}
\item 宇宙線ミューオンは最小電離粒子でありプラスチックシンチレータ中でのエネルギー損失は$dE/dx \sim 2~\mathrm{MeV/cm}$ である.
\item 最頻値をとるのは宇宙線がシンチレータの底面に対して垂直に入射したとき,すなわち1~層を通過する飛跡の長さが6~cm のときである.
\end{itemize}
この仮定から,1~層でのエネルギー損失の最頻値は$2~\mathrm{MeV/cm} \times 6~\mathrm{cm} = 12~\mathrm{MeV}$ に対応すると判断した\cite{leo}.
さらにペデスタルをガウシアンでフィットしてエネルギーの0点に対応する電荷を求めた.図~\ref{ps_langau} は各チャンネルの積分電荷の分布をLangauss 関数 (赤線)でフィッティングしたものである.図~\ref{ps_cali} は宇宙線較正の結果を示したものである.黒色の誤差バー付きの点が較正点,赤色の直線が二点を基に引いた直線である.

\begin{figure}[H]
\centering
\includegraphics[height=0.7\textwidth,angle=-90]{figure/ikemitsu/fit_langau_modify.pdf}
\caption{Langauss 関数を用いた宇宙線ピークのフィッティング.横軸を信号の電荷量の積分値~[pC], 縦軸をカウント数とした.}
\label{ps_langau}
\end{figure} 
\begin{figure}[H]
\centering
\includegraphics[height=0.9\textwidth,angle=-90]{figure/ikemitsu/fit_calibline.pdf}
\caption{プラスチックシンチレータの宇宙線較.横軸を粒子の損失エネルギー,縦軸を対応するエネルギーとした.ペデスタルと宇宙線の測定結果を基に直線を引くことでエネルギー較正を行った.}\label{ps_cali}
\end{figure}

較正の結果は表~\ref{PS_calib_table} のようになった.
表中の$a, b$ は,それぞれエネルギー$E~[\mathrm{MeV }]$ と電荷$Q~[\mathrm{pC}]$ の対応を$E = a Q + b$ としたときの定数である.
\begin{table}[h]
\caption{較正の結果}
\label{PS_calib_table}
\centering
\begin{tabular}{ccc}\toprule
チャンネル番号 & $a$ & $b$ \\ \midrule
0 & 0.40$\pm$0.040 & 0$\pm$0.30 \\
1 & 0.25$\pm$0.030 & 0$\pm$0.06 \\
2 & 0.24$\pm$0.030 & 0$\pm$0.20 \\
3 & 0.34$\pm$0.045 & 0$\pm$0.40 \\
4 & 0.27$\pm$0.035 & 0$\pm$0.07 \\
5 & 0.24$\pm$0.030 &0 $\pm$0.21 \\
6 & 0.39$\pm$0.050 &0 $\pm$0.30 \\
7 & 0.27$\pm$0.040 &0 $\pm$0.08 \\ \bottomrule
\end{tabular}
\end{table}%

\subsubsection{NaI のゲイン測定}
本実験で用いるNaI 検出器が出力する信号は,NaI の9 本とフィンガーカウンターの計10 個であった.一方でNaI の信号の処理に用いるWFD の入力できるチャンネル数は8 つであった.このため,WFD に入力する前にNaI からの信号をアナログの段階で足し合わせて信号の数を絞る必要があった.

各チャンネル間で,NaI 結晶自体の発光量も違い,NaI に付属するPMT のゲインもHV 値(PMT にかける高電圧)に対する依存性が個体毎に大きく異なる.一方で,足し合わせるNaI の信号の大きさは同じエネルギーに対して同じにしなければならかったため,それぞれのNaI のゲインカーブの測定を行った.この測定したゲインからどのようにNaI の信号を足し合わせるかを決めた.

まずNaI のゲインを測定するためにHV 値を変えながら線源${}^{137}\mathrm{Cs}$ を用いて,検出器用に用意した計11本のNaI (1 から順番に11 まで番号を付けた)の電荷の測定を行った(この線源は661.7keV のガンマ線を放出する)\cite{IAEA_ENSDF}.プラスチックシンチレータの宇宙線較正と同様にして,WFD からの電圧信号から電荷分布を得た.これを光電ピークと考えられるところでそれぞれガウシアンでフィッティングし,得られた平均値をゲインとした.さらにゲインについて以下の式が成り立つとし,両対数でフィッティングを行った.\cite{Hamamatsu_PMT} .
\begin{equation}
\mathrm{Gain}~[\mathrm{pC}] = a \times \mathrm{HV}~[\mathrm{kV}]^b \label{gain_curve}  
\end{equation}
ただし$a, b$ はフィッティングパラメータである.その結果が図~\ref{GainHV} である.

また各HV 値でのエネルギー分解能とHV の関係が図~\ref{resoHV} である.ただし分解能はガウシアンの$\sigma$ を電荷で割ったものである.
\begin{figure}[H]
\begin{minipage}{0.45\hsize}
\centering
\hspace*{-1em}
\includegraphics[width=1.1\textwidth]{figure/tajima/gain_curve.eps}
\caption{ゲインとHV 値の対応.横軸をHV 値~[kV],縦軸をゲインに相当する電荷量~[pC]とした.}
\label{GainHV}
\end{minipage}
\hfill
\begin{minipage}{0.45\hsize}
\centering
\includegraphics[width=1.1\textwidth]{figure/tajima/charge_resolution.eps}
\caption{ゲインと分解能の対応.横軸を分解能,縦軸をゲイン~[pC]とした.}
\label{resoHV}
\end{minipage}
\end{figure}
この結果, NaI1 はゲインが低く,NaI5 に関しては分解能が悪かったため,本実験では用いないことにした(以降,NaI1,5 で測定を行わないことにした).
またNaI10 は分解能が良かったため,足し合わせず中心に配置するNaI とした.

上の結果を用いて,ゲインの調整を行った.上のゲインに$50~\mathrm{MeV} /661.7~\mathrm{keV}$ をかけて,スケール変換することで得られた曲線を$50~\mathrm{MeV}$ におけるゲイン曲線とした.そして最大エネルギー$\sim 50~\mathrm{MeV}$ の入力信号の波高~[V] がWFD の入力電圧の上限の半分である$1~\mathrm{V}$ ($500~\mathrm{pC}$) になるように設定した.

表~\ref{HV} が各HV 値の調整値を表にしたものである.ここで入力信号の電圧と電荷の関係は以下のように求めた.まずNaI の波形は立ち上がり時間にくらべ立ち下がり時間が長いので,立ち上がりがステップ関数で立ち下がりが指数関数であることを近似的に仮定した.この時にNaI の信号の減衰時間幅は約$T \sim 250~\mathrm{ns}$ なので,最大電圧$V_\mathrm{max}~[\mathrm{V}]$ の信号電荷は以下のように概算できる.
\begin{equation}
Q~[\mathrm{C}] = \frac{V_\mathrm{max}~[\mathrm{V}] \times T~[\mathrm{s}]}{R~[\Omega]}
\end{equation}
ここで$R$ はWFD の抵抗値で$50~\Omega$ である.

今回測定に用いた線源のエネルギーは661.7~keV であるのに対して,実際の測定の最大エネルギーは約50~MeV であり,線源のエネルギーに比べてかなり大きかった.よって高いエネルギー領域においても上のゲイン曲線が成り立っていることの確認を行った.そのために,表~\ref{HV} のHV 値で宇宙線の測定を行った.そして,得られた電荷分布をLangauss 関数でフィッテイングを行い,得られた最頻値をゲインとすると,表~\ref{nai_gain} のようになった.

このゲインの値が近く,また分解能の値が近いものを足し合わせるペアとし,これを(2,7),(3,9),(4,6),(8,11)の4組に決めた.これらのペアの宇宙線におけるゲインのずれはいずれも10~\%程度となった.

\begin{table}[H]
\begin{minipage}[t]{0.45\textwidth}
\centering
\caption{NaI のHV 設定}\label{HV}
\begin{tabular}{cc}\toprule
NaI No. & HV~[V]\\ \midrule
2 & 1050 \\
3 & 1082 \\
4 & 1125 \\
6 & 1062 \\
7 & 1089 \\
8 & 913 \\
9 & 913 \\ 
10 & 1112 \\
11 & 984 \\ \bottomrule
\end{tabular}
\end{minipage}
\hfill
\begin{minipage}[t]{0.45\textwidth}
\centering
\caption{宇宙線の測定結果}\label{nai_gain}
\begin{tabular}{cc}\toprule
NaI No. & Gain~[pC]\\ \midrule
2 & 2530 \\
3 & 2387 \\
4 & 2352 \\
6 & 2397 \\
7 & 2579 \\
8 & 2423 \\
9 & 2164 \\
11 &2393 \\ \bottomrule
\end{tabular}
\end{minipage}
\end{table}

さらに本実験におけるNaI の配置を表~\ref{haichi} のように決めた.配置を決めるにあたって,以下の事柄を考慮した.
\begin{itemize}
\item エネルギー重心を求める観点からペアを中心から等距離になるように配置した.
\item 中心にフィンガーカウンターを置きコインシデンスをとることで,陽電子が中心のNaI に入射したときのみ測定することにし.このため中心のNaI で測定されるエネルギーは大きくなることが考えられるので,ゲインが低いものを中心に配置した.
\item 本実験のセットアップにおける宇宙線較正を想定した場合(実際には時間の都合上行っていない),測定データから宇宙線がどのNaI にヒットしたか区別できる必要があった.宇宙線は上方から飛来する可能性が高いとして,測定データからどのNaI に宇宙線がヒットしたかを判別できるように配置した.
\end{itemize}
\begin{table}[H]
\centering
\caption{ビーム正面からみたNaI の配置図}\label{haichi}
\begin{tabular}{|c|c|c|}\hline
\cellcolor{yellow}~3~ & \cellcolor{red}2 & \cellcolor{yellow}9\\ \hline
\cellcolor{cyan}~4~ & 10 & \cellcolor{red}7\\ \hline
\cellcolor{green}~8~ & \cellcolor{cyan}6 & \cellcolor{green}11\\ \hline
\end{tabular}
\end{table}

\newpage

\subsubsection{NaI の宇宙線較正}
NaI でエネルギーを測定するために, 宇宙線を用いてエネルギー較正を行った.
先にもとめたHV値のもとで,それぞれのNaI で宇宙線を測定した.
各NaI の上方と下方にプラスチックシンチレータを配置してコインシデンスをとることで,
貫通イベントのみを記録した.
またプラスチックシンチレータの位置をずらすと,測定される宇宙線のNaI 中での飛跡は長くなる.
この時の宇宙線は最小電離粒子とみなせるので,飛跡とエネルギーは比例する.
よって,プラスチックシンチレータの位置をすらすことで計3点のエネルギーを測定した.
そして,得られた電荷分布を,Langauss 関数でフィッテイングを行った.
図~\ref{langau} はあるセットアップにおける宇宙線測定のヒストグラムで,
赤線はLangauss 関数によるフィッテイング曲線である.

次に,同じセットアップで宇宙線モンテカルロシミュレーションを行った.
図~\ref{MC2} はシミュレーションによるエネルギー分布である.
モンテカルロシミュレーションによって得られた最頻値と測定によって得られた最頻値でエネルギー較正を行った.

\begin{figure}[H]
\begin{minipage}{0.45\hsize}
\centering
\hspace*{-1em}
\includegraphics[width=1\textwidth]{figure/tajima/NaI7_Setup2.eps}
\caption{あるセットアップにおける宇宙線測定の結果をLangauss 関数でフィッテイングした結果.横軸を電荷量~[pC],縦軸をカウント数とした.}\label{langau}
\end{minipage}\hfill
\begin{minipage}{0.45\hsize}
\centering
\includegraphics[width=1\textwidth]{figure/tajima/MC2.eps}
\caption{左と同様のセットアップでシミュレーションを行って得られた宇宙線のエネルギースペクトル.横軸をエネルギーを24~MeVで規格化したもの,縦軸をカウント数とした.}\label{MC2}
\end{minipage}
\end{figure}

図~\ref{cali} は宇宙線較正の結果を図にしたものである.
\begin{figure}[H]
\centering
\includegraphics[width=0.7\textwidth]{figure/tajima/fit.eps}
\caption{各ch における宇宙線測定の電荷とエネルギーの対応関係.横軸をエネルギー~[MeV],縦軸をWFD で測定された電荷~[pC]とした.各セットアップにおける測定とシミュレーションの結果を基にした点に対して直線フィッテイングを行った.}\label{cali}
\end{figure}

\newpage

\subsubsection{磁場測定}
$g$ 因子測定のための磁場装置の測定を3 軸テスラメータで行った.銅板を置く範囲を,10~mm 間隔で計63 点測定した.本実験の測定中に磁石が外れてしまったので,磁石が外れる前後のデータとして実験前と実験後に測定した.

測定した結果を以下の表~\ref{MF1}, \ref{MF2} と図~\ref{mag1},\ref{vec1},\ref{mag2},\ref{vec2} に記す.$x, y$の単位は~[mm] で,磁場の単位は~[Gauss] である.原点を銅板の中心とし, $y, z$軸を鉛直方向上向きとビーム方向にとった.
\begin{table}[H]
\centering
\caption{各点の磁場の強さ(磁石が外れる前)}\label{MF1}
\begin{tabular}{|c||c|c|c|c|c|c|c|c|c|}\hline
$y \backslash x$ & -40 & -30 & -20 & -10 & 0 & 10 & 20 & 30 & 40 \\ \hline \hline
30 & 51.48 & 54.98 & 55.33 & 55.55 & 55.42 & 54.77 & 54.44 & 53.27 & 51.75 \\ \hline
20 & 56.25 & 56.78 & 56.87 & 56.84 & 56.41 & 56.16 & 55.93 & 55.71 & 55.35 \\ \hline
10 & 57.15 & 57.50 & 57.39 & 57.11 & 56.72 & 56.53 & 56.48 & 56.50 & 56.44 \\ \hline
0 & 57.60 & 57.49 & 57.19 & 56.72 & 56.54 & 56.48 & 56.53 & 56.58 & 55.98 \\ \hline
-10 & 56.66 & 56.78 & 56.66 & 56.48 & 56.39 & 56.34 & 56.35 & 56.31 & 56.12 \\ \hline
-20 & 53.45 & 54.74 & 55.29 & 55.54 & 55.60 & 55.60 & 55.53 & 55.14 & 54.3 \\ \hline
-30 & 50.62 & 51.95 & 53.45 & 53.93 & 54.20 & 54.20 & 54.05 & 53.11 & 51.25 \\ \hline
\end{tabular}
\end{table}
\begin{figure}[H]
\centering
\begin{minipage}{0.45\hsize}
\centering
\includegraphics[width=1\textwidth]{figure/tajima/mag1.eps}
\caption{磁場の強さの分布図(磁石が外れる前).横軸を$x$,縦軸を$y$ とした.黒線に囲われた領域が銅板領域に対応する.}
\label{mag1}
\end{minipage}
\begin{minipage}{0.45\hsize}
\centering
\includegraphics[width=1\textwidth]{figure/tajima/vec1.eps}
\caption{xy平面における磁場ベクトル(磁石が外れた後).横軸を$x$,縦軸を$y$ とした.背面の色は$z$ 軸方向の磁場の大きさを表し,矢印の長さは磁場ベクトルの大きさに対応している.}
\label{vec1}
\end{minipage}
\end{figure}



\begin{table}[H]
\centering
\caption{各点の磁場の強さ(磁石が外れた後)}\label{MF2}
\begin{tabular}{|c||c|c|c|c|c|c|c|c|c|}\hline
$y \backslash x$ & -40 & -30 & -20 & -10 & 0 & 10 & 20 & 30 & 40 \\ \hline \hline
30 & 55.41 & 55.93 & 55.79 & 54.78 & 53.30 & 50.73 & 46.89 & 46.34 & 41.95 \\ \hline
20 & 57.25 & 57.06 & 56.28 & 54.92 & 53.77 & 52.01 & 49.83 & 48.45 & 48.44 \\ \hline
10 & 57.72 & 57.30 & 56.52 & 55.61 & 54.34 & 52.97 & 51.84 & 51.08 & 50.49 \\ \hline
0 & 57.55 & 56.94 & 56.30 & 55.54 & 54.61 & 53.67 & 52.98 & 52.40 & 52.34 \\ \hline
-10 & 57.00 & 56.57 & 55.94 & 55.24 & 54.58 & 53.99 & 53.45 & 53.05 & 52.73 \\ \hline
-20 & 55.48 & 55.37 & 55.04 & 54.64 & 54.30 & 53.83 & 53.39 & 52.53 & 51.81 \\ \hline
-30 & 52.23 & 53.03 & 53.28 & 53.29 & 53.18 & 52.71 & 52.00 & 51.51 & 50.12 \\ \hline
\end{tabular}
\end{table}
\begin{figure}[H]
\centering
\begin{minipage}{0.45\hsize}
\centering
\includegraphics[width=1\textwidth]{figure/tajima/mag2.eps}
\caption{磁場の強さの分布図(磁石が外れた後).黒線に囲われた領域が銅板領域に対応する.左手奥の磁石が外れ,隣の磁石についてしまった.これにより左部に磁場の偏りがみられる.}
\label{mag2}
\end{minipage}
\begin{minipage}{0.45\hsize}
\centering
\includegraphics[width=1\textwidth]{figure/tajima/vec2.eps}
\caption{$xy$ 平面における磁場ベクトル(磁石が外れた後).背面の色は$z$ 軸方向の磁場の大きさを表し,矢印の長さは磁場ベクトルの大きさに対応している.}
\label{vec2}
\end{minipage}
\end{figure}

$g$ 因子の計算に用いる磁場を得るために,予め頂いたビームプロファイルのデータ($\sigma_x = 33.3037~\mathrm{mm}, \; \sigma_y = 19.6668~\mathrm{mm}$) を用いて加重平均をとった.下図~\ref{beam_mag}が加重平均に用いた重み分布で,平均$(\mu_x,\mu_y)=(0,0)$ ,分散$\sigma_x^2, \sigma_y^2$の2次元のガウシアンを規格化したものである.加重平均の結果,磁石が外れる前後の磁場がそれぞれ$56.06 \pm 0.01~\mathrm{G}$ と$53.97\pm 0.01~\mathrm{G}$ と求まった.誤差は統計誤差で,各点の測定の測定誤差を伝播させることで求めた.(系統誤差については後述.)

\begin{figure}[H]
\centering
\includegraphics[width=0.5\textwidth]{figure/tajima/beam_mag.eps}
\caption{加重平均に用いたビームプロファイル.黒線に囲われた領域が銅板領域に対応する.}
\label{beam_mag}
\end{figure}

\newpage

\subsection{ビームを用いた本実験}

\subsubsection{タイムスケジュール}
本実験におけるスケジュールは以下の通りである.
\begin{itemize}
\item 2/25 (Sun.)\\
13:00   東海村到着,前日準備
\item 2/26 (Mon.) - 2/27 (Tue.)\\
9:00 - ロシアグループ $g - 2$ Beam Profile Monitor の実験の傍らで寿命を測定,セットアップの確認, 磁場測定(磁石が外れる前)
\item 2/28 (Wed.)\\
12:30  \phantom{-} ロシアグループの実験終了,銅板標的にて実験開始\\ 
12:30 - 21:30 セットアップの確認\\
21:47 - 24:00 磁場ターゲットを置いて,$g$ 因子をNaI のみで測定\\
24:20 - 29:50 磁場ターゲットを置いて,$g$ 因子をPS を加えて測定(測定の途中で磁石が外れる)
\item 3/1(Thu.)\\
7:20 \phantom{-} 磁場測定(磁石が外れた後)\\
7:20 - 8:10 銅板標的を置いて,エネルギーと寿命を測定\\
8:30 \phantom{-} ビームタイム終了\\
8:30 - 14:30 実験装置の放射線チェック及び後片付け \\
14:30 帰宅
\end{itemize}

\subsubsection{セットアップ}
この節では本実験のセットアップについて記す.まず寿命測定のセットアップは図~\ref{set_life}, \ref{set_life2} のようであった.図~\ref{set_life} が上からみたセットアップ図で図~\ref{set_life2} が実際のセットアップの写真である.ここでFC ,PS はフィンガーカウンターとプラスチックシンチレータの略で,Target は先に述べた銅板標的のことである.標的から測定器までの距離をビーム1~pulse あたりの陽電子のカウント数から決定した.

今回の測定では解析の都合から,陽電子がNaI では1~pulse あたり5 - 6 個に,プラスチックシンチレータでは8 - 9 個になるように調整を行った.その結果,標的からNaI ,プラスチックシンチレータまでの距離はそれぞれ$115~\mathrm{cm}, \; 75~\mathrm{cm}$ ,標的の中心からNaI ,プラスチックシンチレータまでの直線とビーム方向のなす角はそれぞれ$41^{\circ}, \;  47^{\circ}$ となった.
また標的とビーム窓の距離は17~cm であった.
\begin{figure}[H]
\begin{minipage}{0.45\hsize}
\centering
\includegraphics[width=1\textwidth]{figure/tajima/set_lifetime.png}
\caption{寿命測定のセットアップ}
\label{set_life}
\end{minipage}
\begin{minipage}{0.45\hsize}
\centering
\includegraphics[width=1\textwidth]{figure/tajima/set_lifetime_1.png}
\caption{寿命測定のセットアップ(写真)}
\label{set_life2}
\end{minipage}
\end{figure}

次にエネルギー,$g$ 因子測定のセットアップについて示す.図~\ref{set_life} が上空からみたセットアップ図で,図~\ref{set_g_2} が実際のセットアップの写真である.ここでTarget には先に記した磁場印加標的を用いた.また検出器と標的の距離やビーム軸と標的から検出器のなす角は寿命測定と同じ値になるようにした.
\begin{figure}[H]
\centering
\includegraphics[width=0.5\textwidth]{figure/tajima/g.jpg}
\caption{$g$ 因子測定のセットアップ(写真)}
\label{set_g_2}
\end{figure}

\subsubsection{回路}
この節では本実験の回路について述べる.図~\ref{cir_PS}, \ref{cir_nai}はNaI ,プラスチックシンチレータの回路である.ここでBeam Trigger とは,MLF 施設からの信号のことで,時刻の基準(TO)となっている.これをWFD の外部トリガーとして接続した.また図~\ref{cir_PS}の$n$ - $s \; (n = 1, 2, 3, 4,  s = a, b)$ はプラスチックシンチレータの$n$ 層目のファイバーの片側から読み出される信号を意味する.図中に記載している端子名 (BNC,LEMO,MCX) は接続したケーブルの両端の端子名である.FADC は Waveform Digitizer のことである.
\begin{figure}[H]
\begin{minipage}{0.45\hsize}
\centering
\includegraphics[width=1\textwidth]{figure/tajima/circuit_ps_2.png}
\caption{プラスチックシンチレータ検出器用の回路}
\label{cir_PS}
\end{minipage}
\hfill
\begin{minipage}{0.45\hsize}
\centering
\includegraphics[width=1\textwidth]{figure/tajima/circuit_nai.png}
\caption{NaI 検出器用の回路}
\label{cir_nai}
\end{minipage}
\end{figure}
本実験ではコリメータだけでは不十分で,プラスチックシンチレータにおいても標的から飛んできた陽電子かどうかを判定するためのフィンガーカウンターを設置する必要性が浮上したため,コリメータ前方に設置した.その際にWFD の入力数が不足したため,中間層でありデータとして価値が低いであろうと考えたプラスチックシンチレータの3 層目を片側読み出しにすることにした.

\subsubsection{実験手順}
本実験は以下の手順で行った.
\begin{enumerate}
\item 実験セットアップの調整.
\item ビームが照射される領域に人がいないことを確認し,人が侵入することのないように施錠.
\item ビーム輸送装置上のコリメータを開放し,ビームが標的に照射.
\item コリメーターを制御することで,ビーム1パルス当たりのレートを調整.必要な場合はセットアップを調整.
\item データテイキングを開始.
\end{enumerate}

%\end{document}

%\documentclass[titlepage]{jsarticle}

%\usepackage[dvipdfmx]{graphicx}

%数式用パッケージ
%\usepackage{amsmath, amssymb}
%\usepackage{mathtools}
%\usepackage{cancel}
%\usepackage{cases}
%\usepackage{bm}
%\usepackage{array,booktabs}
%\usepackage{float}

%\usepackage{subcaption}

%ファインマングラフ用パッケージ
%\usepackage{feynmf}



%\graphicspath{{./figure/}}

%\begin{document}


%%%%%%%%%%%%%%%%%%%池満パート開始%%%%%%%%%%%%%%%%%%%
 \section{プラスチックシンチレータで取得したデータの解析と考察}
 プラスチックシンチレータのデータの解析では波形解析やイベント選択の手法の異なる2つの解析手法を行った.それぞれで寿命と$g$因子の2つを求めた.
 以下ではそれぞれに解析手法Aと解析手法Bと名付けた.また,最後に両方の結果をまとめた.
 
  \subsection{解析手法A}
  \subsubsection{解析の流れ}
  プラスチックシンチレータ(以下,PS)検出器を用いて取得したデータを以下の方法で解析し,寿命,$g$因子,エネルギー分布を求めた.
  \begin{enumerate}
   \item イベントディスプレイから,初めの100~ns (50~Sample)の間は信号が来ていないことを確認し,0〜100~ns のデータの平均値をとってそれをbaselineとした.
   \item 信号のしきい値(threshold)の決定
	 \begin{itemize}
	  \item 宇宙線を用いた予備実験の結果から,12~MeV のエネルギーに対応する信号のピーク値を求めた.(表~\ref{12MeVpeak})%宇宙線を用いた予備実験の説明は?? 12MeVがMIPがPSの厚さを通った時に落とすエネルギーである説明.
	  \item そのピーク値から,各チャンネルごとのthresholdを以下の値に決めた.
		ただし,本実験ではチャンネル4でfingerの信号を取ったので,fingerのthresholdは表~\ref{12MeVpeak}と関係無くイベントディスプレイを元に適当な値(20~mV)に決めた.
	  \item 寿命測定と$g$因子測定用には全チャンネルとも4~MeV 相当とした.4~MeV 相当にした理由については後述する.
	  \item エネルギー測定用には,1層目にのみ4~MeV 相当のthresholdを設けた.2層目以降はthresholdを設けなかったのは,4~MeV 相当のthresholdを超えるエネルギーを持つイベントを全て取るためである.
	  %\item エネルギー測定用には,1層目:4~MeV 相当とし,2層目以降はthresholdを設けなかった.%なぜ?
	 \end{itemize}
   \item 各チャンネルごとで,thresholdを越えた時間を信号の時間(peaktime)とした.
   \item イベントディスプレイからおおよその信号の時間幅を決め,信号が検出されてから次の信号を検出するようになるまでのveto時間を40~ns にした.
   \item peaktimeから40~ns の間のデータを足すことで信号のchargeを求めた.
   \item 寿命と$g$因子について
	 \begin{itemize}
	  \item 各層の両側のチャンネルの信号のcoincidenceを取った.ただし,3層目は片側のみの信号である.
	  \item ここで,coincidenceの条件は互いのpeaktimeが10~ns よりも近いものとした.
	  \item 寿命測定ではフィンガーカウンター(以下,finger)を要求せずに層ごとのcoincidenceのみをとった.
	  \item $g$因子測定では立体角を制限するために,層ごとだけでなくfingerとのcoincidenceを要求した.
	 \end{itemize}
   \item エネルギーについて
	 \begin{itemize}
	  \item 予備実験のデータから,各チャンネルごとにキャリブレーションをした.%上に同じ,実験説明が無い.
	  \item 各層のエネルギーとして,1,2,4層目では両側のチャンネルのエネルギーの平均を取り,3層目では片方のチャンネルのエネルギーを使用した.
	  \item fingerと1層目の両側のチャンネルの信号のcoincidenceをとり,そのときの全層のエネルギーの和を求めた.
	  \item そのエネルギーにfingerで落とすエネルギーとして1.5~MeV を足した.
	  \item fingerとのcoincidenceを取ったのは,検出器の中心に入ったe$^{+}$の信号のみを選択するためである.
		高エネルギーのe$^{+}$が入射すると検出器との相互作用によって電磁シャワーが生じる.
		シャワー中の粒子が検出器内で反応せずに外に漏れ出ると,検出されるエネルギーは実際のe$^{+}$のエネルギーよりも低い値になる.
		e$^{+}$の入射位置が検出器の端にある場合,検出器中央に入射する場合よりも電磁シャワー中の電子や陽電子が検出器の外に漏れやすい.
		以上のことからe$^{+}$が検出器の中央に入射したイベントのみを解析に用いるべきだと考え,fingerとのcoincidenceを要求した.
	  %\item fingerとのcoincidenceを取ったのは,検出器の中心に入ったe$^{+}$の信号のみを選択し,エネルギー漏れを減らすためである.%エネルギー漏れとは?
	 \end{itemize}
  \end{enumerate}

  寿命と$g$因子の解析に関して,解析方法Bとの主な違いは次の3点である.
  \begin{itemize}
   \item thresholdの値を各チャンネル毎に変えていること
   \item 信号検出後にveto時間を設けていること
   \item 層間のcoincidenceやfingerとのcoincidenceをとっていること
  \end{itemize}
  
  \begin{table}[h]
   \caption{12~MeV に対応するchargeをもつ信号のピーク}
   \label{12MeVpeak}
  \begin{center}
   \begin{tabular}{cc}\toprule
    PMT名&ピーク値~(mV) \\ \hline
    チャンネル0(あけみ)&110 $\pm$ 21 \\
    チャンネル1(勝太郎)&145 $\pm$ 25 \\
    チャンネル2(畑さん)&160 $\pm$ 28 \\
    チャンネル3(紗智子)&112 $\pm$ 22 \\
    チャンネル4(蘭)  &140 $\pm$ 23 \\
    チャンネル5(矢部) &161 $\pm$ 28 \\
    チャンネル6(政子) &105 $\pm$ 21 \\
    チャンネル7(王)  &140 $\pm$ 25 \\ \bottomrule
   \end{tabular}
  \end{center} 
  \end{table}


  \subsubsection{信号検出のthreshold値について}
  信号検出時のthresholdの値の判断理由について触れておく.
  
  thresholdの値を1~MeV 相当から6~MeV 相当まで変化させて寿命を求めると,thresholdが1~MeV 相当から高くなるにつれて寿命は短くなり,4~MeV 相当以上のときに誤差の範囲で一定の値になった.
  このことから3~MeV 相当以下のthresholdでは低エネルギーのノイズを信号として処理していると考えた.
  よって,thresholdの値には4~MeV 相当の電圧値が妥当だと考えた.
  %1層目のthresholdを他層よりも高く設定しているのは,このようなバックグラウンドを除去するためである.%なぜ除去できる?
  %一方,寿命測定に関して,thresholdを上げてもfittingの結果は変わらず,統計誤差が大きくなるだけだった.%表現に違和感

  \subsubsection{得られた崩壊曲線とfittingの結果}
  図~\ref{lt_layercoin}は,磁場なし標的を用いたときのミューオンの崩壊曲線である.
  それを次の$f_{\mathrm{life}}(t)$でfittingした結果が図~\ref{lt_layercoin_fit}である.
  \begin{equation*}
   f_{\mathrm{life}}(t) = \exp[-(t+A)/\tau].
  \end{equation*}
  また,図~\ref{g_layercoin}は磁場あり標的を用いたときのミューオンの崩壊曲線である.
  それを次の$f_{g}(t)$でfittingした結果が図~\ref{g_layercoin_fit}である.
  \begin{equation*}
   f_{g}(t) = \exp[-(t+A)/\tau](1+B\cos(\omega t + \delta)).
  \end{equation*}
  
  fittingの範囲は1050~ns から7950~ns までとした. 
  今回使用したミューオンビームの分布はFWHM$\sim$100~ns ,すなわち$\sigma\sim$50~ns のガウス分布と考えることができる.
  図~\ref{lt_layercoin}のヒストグラムがピーク値をとる約800~ns をガウシアンのピークだと考えて,その点から$5\sigma\sim 250$~ns はfittingの対象外とした.
  また,ピークを検出してから40~ns のveto時間をとるという解析方法から,8000~ns から後ろに40~ns 分は解析に用いることができない.
  以上のことからfitting範囲を決定した.
  
  fittingの結果は表~\ref{fit_lt},~\ref{fit_g}のようになった.表中の誤差はfittingに由来する統計誤差である.
  
  \begin{table}[H]
   \caption{寿命$\tau$のfitting結果}
   \label{fit_lt}
   \begin{center}
    \begin{tabular}{cc}\toprule
     coincidenceを取った層& $\tau$~(ns) \\ \midrule
     1 		        & 2223.7 $\pm$ 7.7 \\
     1+2 		& 2224 $\pm$ 10 \\
     1+2+3 		& 2217 $\pm$ 15 \\
     1+2+3+4 		& 2148 $\pm$ 35\\ \bottomrule
    \end{tabular}
   \end{center}
  \end{table}%
  
  \begin{table}[H]
   \caption{$g$因子のfitting結果}
   \label{fit_g}
   \begin{center}
    \begin{tabular}{cccc}\toprule
     coincidenceを取った層&$\tau$~(ns)& $\omega$~($\times 10^{-3}$/ns) & $g$ \\ \midrule
     finger+1             &2170.4 $\pm$ 9.1 & $( 4.621 \pm 0.012 ) $ & 2.0091 $\pm$ 0.0054 \\
     finger+1+2 	  &2199 $\pm$ 13    & $( 4.614 \pm 0.011 ) $ & 2.0063 $\pm$ 0.0050 \\
     finger+1+2+3 	  &2205 $\pm$ 18    & $( 4.594 \pm 0.013 ) $ & 1.9973 $\pm$ 0.0055\\
     finger+1+2+3+4 	  &2142 $\pm$ 45    & $( 4.649 \pm 0.024 ) $ & 2.021 $\pm$ 0.011 \\ \bottomrule
    \end{tabular}
   \end{center}
  \end{table}%


  \begin{figure}[H]
   \centering
   \begin{subfigure}{\columnwidth}
    \centering
    \includegraphics[height = 0.9\columnwidth , angle = -90]{figure/ikemitsu/lt_layercoin.pdf}
    \caption{層でcoincidenceを取って得られたヒストグラム(磁場なし標的)}
    \label{lt_layercoin}
   \end{subfigure}
   \begin{subfigure}{\columnwidth}
    \centering
    \includegraphics[height = 0.9\columnwidth , angle = -90]{figure/ikemitsu/lt_layercoin_fit.pdf}
    \caption{$f_{\mathrm{life}}(t)$でfittingをした図}
    \label{lt_layercoin_fit}
   \end{subfigure}
   \caption{fittingした図}
   \label{lt_layercoin_all}
  \end{figure}
  
  \begin{figure}[H]
   \centering
   \begin{subfigure}{\columnwidth}
    \centering
    \includegraphics[height = 0.9\columnwidth , angle = -90]{figure/ikemitsu/g_layer_f_coin.pdf}
    \caption{fingerと層でcoincidenceを取って得られたヒストグラム(磁場あり標的)}
    \label{g_layercoin}
   \end{subfigure}
   \begin{subfigure}{\columnwidth}
    \centering
    \includegraphics[height = 0.9\columnwidth , angle = -90]{figure/ikemitsu/g_layer_f_coin_fit.pdf}
    \caption{$f_{g}(t)$でfittingをした図}
    \label{g_layercoin_fit}
   \end{subfigure}
   \caption{fittingした図}
   \label{g_layercoin_all}
  \end{figure}

  
    \subsubsection{fitting範囲の妥当性について}
  上で述べたようにfittingの範囲は1050~ns から7950~ns としたが,この範囲の前側と後側で範囲を分けてfittingをすると,$\tau$と$g$の値は表~\ref{fitrange1}〜~\ref{fitrange4}のようになった.
  図~\ref{lt_diff}と図~\ref{g_diff}はそれぞれ,表~\ref{fitrange1}・表~\ref{fitrange2}・表~\ref{fit_lt}と表~\ref{fitrange3}・表~\ref{fitrange4}・表~\ref{fit_g}をグラフにしたものである.
  図中の黒丸が1050~ns〜7950~nsでのフィット結果,赤四角が1050~ns〜4550~nsでのフィット結果,青三角が4450~ns〜7950~nsでのフィット結果を表していて,左から順に1層のみ,1・2層coincidence,1〜3層coincidence,全層coincidenceのときの値である.
  各層毎で3つのfitting範囲の結果を比較すると,$\tau$と$g$の値は誤差の範囲内で一致していると言える.
  このことから,fittingの範囲として1050~ns から7950~ns までは適切だと考えられる.
  %よって,図~\ref{lt_layercoin}のヒストグラムのピーク点は750~ns であるが,その点から1000~ns はfittingの対象外とし,%なぜ1000ns?
  %図~\ref{lt_layercoin_fit}と図~\ref{g_layercoin_fit}ではfittingの範囲を1750~ns から7950~ns までとした.
  %寿命に関して全層のcoincidenceがfittingの範囲が前半の時と後半の時とではコンシステントではない.系統誤差となるのでは?

  \begin{figure}[H]
   \centering
   \begin{minipage}{0.4\columnwidth}
    \centering
    \includegraphics[height = \columnwidth,angle=-90]{figure/ikemitsu/lt_difference.pdf}
    \caption{fitting範囲別の寿命の値}
    \label{lt_diff}
   \end{minipage}
   \begin{minipage}{0.4\columnwidth}
    \centering
    \includegraphics[height=\columnwidth,angle=-90]{figure/ikemitsu/g_differnce.pdf}
    \caption{fitting範囲別の$g$の値}
    \label{g_diff}
   \end{minipage}
  \end{figure}
  
  \begin{table}[H]
   \centering
   \begin{minipage}{0.4\columnwidth}
    \caption{寿命$\tau$;fitting範囲1050~ns 〜4550~ns }
    \label{fitrange1}
    \begin{center}
     \begin{tabular}{cc}\toprule
      coincidenceを取った層 & $\tau$~(ns) \\ \midrule
      1 			   & 2233 $\pm$ 14 \\
      1+2 		   & 2234 $\pm$ 19 \\
      1+2+3 		   & 2249 $\pm$ 27 \\
      1+2+3+4 	           & 2127 $\pm$ 63 \\ \bottomrule
     \end{tabular}
    \end{center}
   \end{minipage}
   \hspace*{5mm}
   \begin{minipage}{0.4\columnwidth}
    \caption{寿命$\tau$;fitting範囲4450~ns 〜7950~ns }
    \label{fitrange2}
    \begin{center}
     \begin{tabular}{cc}\toprule
      coincidenceを取った層 & $\tau$~(ns) \\ \midrule
      1 			   & 2250 $\pm$ 32 \\
      1+2 		   & 2238 $\pm$ 42 \\
      1+2+3 		   & 2257 $\pm$ 61 \\
      1+2+3+4 		   & 2013 $\pm$ 121 \\ \bottomrule
     \end{tabular}
    \end{center}   
   \end{minipage}
  \end{table}%

  \begin{table}[H]
    \caption{$g$因子;fitting範囲1050~ns 〜4550~ns }
    \label{fitrange3}
    \begin{center}
     \begin{tabular}{cccc}\toprule
      coincidenceを取った層&$\tau$~(ns)& $\omega$~($\times 10^{-3}$/ns) & $g$ \\ \midrule
      finger+1             &2134 $\pm$ 15 & $( 4.638 \pm 0.020 ) $ & 2.0165 $\pm$ 0.0088 \\
      finger+1+2 	  &2185 $\pm$ 21 & $( 4.612 \pm 0.020 ) $ & 2.0054 $\pm$ 0.0085 \\
      finger+1+2+3 	  &2221 $\pm$ 32 & $( 4.615 \pm 0.021 ) $ & 2.0068 $\pm$ 0.0093\\
      finger+1+2+3+4 	  &2271 $\pm$ 87 & $( 4.663 \pm 0.040 ) $ & 2.028 $\pm$ 0.018 \\ \bottomrule
     \end{tabular}
    \end{center}    
  \end{table}%

  \begin{table}[H]
   \caption{$g$因子;fitting範囲4450~ns 〜7950~ns }
   \label{fitrange4}
   \begin{center}
    \begin{tabular}{cccc}\toprule
     coincidenceを取った層&$\tau$~(ns)& $\omega$~($\times 10^{-3}$/ns) & $g$ \\ \midrule
     finger+1             &2254 $\pm$ 42 & $( 4.569 \pm 0.059 ) $ & 1.986 $\pm$ 0.026 \\
     finger+1+2 	  &2267 $\pm$ 58 & $( 4.618 \pm 0.053 ) $ & 2.008 $\pm$ 0.023 \\
     finger+1+2+3 	  &2245 $\pm$ 86 & $( 4.571 \pm 0.056 ) $ & 1.987 $\pm$ 0.024\\
     finger+1+2+3+4 	  &1975 $\pm$ 186& $( 4.64 \pm 0.11 ) $ & 2.016 $\pm$ 0.046 \\ \bottomrule
    \end{tabular}
   \end{center}
  \end{table}%

  \subsubsection{coincidenceをとる層数による寿命の違いについて}
  図~\ref{lt_diff}において,各fit範囲毎に全層coincidence以外の3点を定数でfittingすると図~\ref{lt_diff_fit}のようになった.
  全層でcoincidenceをとったときのfitting結果が3層目までの結果から明らかにずれていることがわかる.
  このずれが何に起因しているのかはわからなかったので,全層coincidenceの結果は系統誤差に含めることにした.

  \begin{figure}[H]
   \centering
   \includegraphics[height=0.6\columnwidth,angle=-90]{figure/ikemitsu/lt_difference_fit.pdf}
   \caption{各fit範囲毎に,全層coincidence以外の3点を定数でfittingした様子}
   \label{lt_diff_fit}
  \end{figure}

  $g$因子に関しても同様に,図~\ref{g_diff}において全層coincidence以外の3点を定数でfittingした.
  表~\ref{fit_lt}と表~\ref{fit_g}の値に対して行なったこれらの定数fittingの結果を解析手法Aの結果として表~\ref{kaisekiA_matome}にまとめた.
  誤差は,3点の定数fittingの誤差を統計誤差とし,全層coincidenceで得られた値の中央値との差を系統誤差とした.

  \begin{table}[H]
   \caption{解析手法Aで得た寿命と$g$因子の結果}
   \label{kaisekiA_matome}
   \begin{center}
    \begin{tabular}{cc}\toprule
     寿命$\tau$~(ns)       &$g$因子 \\ \hline
     $2222.8\pm5.7 _{-75}$ &$2.0044\pm 0.0031^{+0.017}$ \\ \bottomrule
    \end{tabular}
   \end{center}
  \end{table}
  
  \subsubsection{エネルギー分布}
  磁場なし標的を用いたときのデータから求めたエネルギーのヒストグラムは図~\ref{michel_PS}のようになった.
  ミューオンビームのピークと考えられる時間(800~ns 付近)から約$5\sigma$~(250~ns)の間隔を開けることでミューオンが散乱されて直接検出器に入射するイベントはほぼ無視できると考えた.
  したがって解析にはpeaktimeが1050~ns 以降のイベントのみを使用した.
  
  図~\ref{michel_PS}において横軸と縦軸の誤差として,それぞれキャリブレーション由来のエネルギー分解能と統計誤差をつけたグラフが図~\ref{michel_PS_gosa}である.
  %磁場なし標的を用いたときのデータから求めたエネルギー分布は図~\ref{michel_PS}のようになった.
  %さらに,~\ref{michel_PS}の各点において,キャリブレーションに由来するエネルギー分解能と統計誤差をそれぞれ横軸と縦軸の誤差として付けたのが図~\ref{michel_PS_gosa}である.%図の縦軸の説明 2つの図で全然スケールが違うのはなぜ?
  \begin{figure}[H]
   \centering
   \begin{minipage}{0.4\columnwidth}
    \centering
    \includegraphics[height=\columnwidth,angle=-90]{figure/ikemitsu/michel_PS.pdf}
    \caption{PSで得られたエネルギー分布;\\磁場なし標的}
    \label{michel_PS}    
   \end{minipage} 
   \begin{minipage}{0.4\columnwidth}
    \centering
    \includegraphics[height=\columnwidth,angle=-90]{figure/ikemitsu/michel_PS_gosa.pdf}
    \caption{PSで得られたエネルギー分布;\\磁場なし標的(誤差付き)}
    \label{michel_PS_gosa} 
   \end{minipage}
  \end{figure}
  
  ミューオンのスピンに対して角度$\theta$の方向に崩壊する$\mathrm{e}^{+}$のエネルギー分布は,式~\ref{eq:theory_michel}で与えられた.
  この式に$\rho$以外のパラメータの値として,標準模型で予想されている$\eta = 0 , \xi = 1 , \xi \delta = 3/4$を代入して計算すると,
  \begin{equation*}
   \frac{d\Gamma}{dx} \propto x^{2} [\frac{2}{3}(\rho + \frac{3}{8}\cos \theta - \frac{1}{8})(4x-3) + \frac{1}{4}(\cos \theta + 3)]
  \end{equation*}
  となる.

  図~\ref{michel_PS_gosa}のグラフを$f_{\mathrm{michel}}(x) = x^{2} (A(4x -3) + B)$でfittingすることにより$\rho$を求めることができると考えたが,fittingはうまくいかなかった.
  この原因として,高エネルギー側での横軸の誤差が大きいことや検出器のエネルギー分解能を考慮せずにfittingを行っていることが考えられる.
  本解析では時間の都合上これ以上の解析は行なわなかった.

  \subsubsection{エネルギー測定についての今後の課題}
  以下ではエネルギー分布測定についての今後の方針を述べる.

  PSは無機シンチレータに比べて密度が小さいので制動放射で生じるフォトンは少ない.
  したがってフォトンによるエネルギー漏れは少なく,これに起因する測定エネルギーのゆらぎは比較的小さい.
  しかし,シンチレーション効率が低いPSの信号をPMTで読み出すにはゲインを高くする必要があり,PMTの電子増倍過程の確率的なゆらぎによってエネルギー分解能が悪くなる.
  %PSは無機シンチレータに比べて密度が小さいので,電磁シャワーで生じたフォトンが検出器の外側に漏れやすい.%フォトンの発生の頻度がNaIに比べて低いことも注目すべき
  %これによって測定されるエネルギーが入射粒子のエネルギーよりも低くなるイベントが生じる.
  %したがって測定の結果得られるエネルギー分布は入射粒子の分布と比較して高エネルギー側が少なく低エネルギー側が多い分布になる.
  %これによって高エネルギーの粒子の観測数が減ると考えられる.%観測数とはなにか?
  したがって,解析においては測定されるエネルギーにPMT由来のゆらぎがあることを考慮する必要がある.
  %解析的にこの課題を解決するには,シミュレーションによって入射粒子のエネルギーと観測されるエネルギーの対応を求めてfitting関数にその寄与を組み込む必要がある.

  具体的には,まず,エネルギー$E_\mathrm{in}$をもつ粒子が検出器中で失うエネルギー$E_\mathrm{loss}$の分布をシミュレーションによって求め,それを適当な関数で近似する.
  これを0から50~MeV の$E_\mathrm{in}$に対して行い,$E_\mathrm{loss}$の分布関数$f(E_\mathrm{in},E_\mathrm{loss})$を求める.
  次に,入射粒子のエネルギー分布$g_{\mathrm{real}}(E)$を$f(E_\mathrm{in},E_\mathrm{loss})$でたたみ込み積分することで,損失エネルギーの分布関数$g_{\mathrm{loss}}(E)$が得られる.
  最後に,PMTに起因するゆらぎの寄与として$g_{\mathrm{loss}}(E)$をガウシアンでたたみ込み積分して,予想される測定値の分布関数$g_{\mathrm{measure}}(E)$が得られる.
  $g_{\mathrm{measure}}(E)$を用いてfitting行う.
  %解析的にこの課題を解決するには,シミュレーションによって入射粒子のエネルギーと観測されるエネルギーの対応を求めてfitting関数にその寄与を組み込む必要がある.%それでそれに対してどうするのか?
  
  また,この実験ではPSのエネルギー較正に改善の余地が大いに残されている.
  一つの方法としてベータ線源の$Q$~値を求めることで較正点を増やすことができる.
  他に,宇宙線ミューオンを用いたキャリブレーションならば,fingerとのcoincidenceを取ることによって宇宙線の飛跡を制限して測定を行うことが考えられる.
  こうすることで測定される宇宙線の損失エネルギーの幅は図~\ref{ps_langau}よりも狭くなり,また,いくつかの飛跡に対してエネルギー測定を行うことで異なる損失エネルギーに対する電荷量を求められる.
  すなわち,較正点が増えて各点における誤差が小さくなると予想される.
  したがって,測定する宇宙線の飛跡を制限することで較正直線の誤差を小さくすることができると考えられる.
  %宇宙線ミューオンを用いたキャリブレーションならば,NaI検出器で行っているような,fingerとのcoincidenceをとることで精度を上げることができると考えられる.
  % NaIではfingerとのcoincidenceをとるといった処理はしていない.事実誤認.

  %%%%%%%%%%%%%%%%%%%%%%%%%%%%%%%%%%%%%%%%%%%%%%%%%%%%%
  %しかし,寿命に関して表~\ref{fitrange1}と表~\ref{fitrange2}の1,2段目を比較すると,遅い時間側でfittingを行うと寿命が長くなっていると分かる.
  %これは,バックグラウンドなどのノイズを信号として処理しており,その影響がイベント数の少ない部分で強く出ていることによると考えることができる.
  %これを除くためには,より詳しい波形解析によってノイズとみなせる信号の性質を特定しなければならない.
  %今回の実験ではバックグラウンド計測をしなかったが,今後の課題として,ノイズ除去を目的とした解析を行う必要がある.
  %ノイズとして想定されるものには,
  %WFDを用いた実験では波形をデータとして残せるので,こうしたノイズに起因する信号を取り除くことは可能だと考えられる.
  %寿命の解析結果として,表~\ref{fit_lt}と表~\ref{fitrange1}・表~\ref{fitrange2}の違いをfitting範囲による系統誤差に含めた.%どのように?
  %WFDを用いた実験では波形をデータとして残せるので,ノイズに起因する信号を取り除くことは可能だと考えられる.%どのようなノイズが想定されている?
  %寿命の解析結果として,表~\ref{fit_lt}と表~\ref{fitrange1}・表~\ref{fitrange2}の違いをfitting範囲による系統誤差に含めた.%どのように?

  %$g$因子については,バックグラウンドがあっても振動の周期への影響はないと考えられる.%なぜ?
  %したがって,fitting範囲による系統誤差はないとした.

  %%%%%%%%%%%%%%%%%%%池満パート終了%%%%%%%%%%%%%%%%%%%

 %\end{document}

%\documentclass{jsarticle}

%\usepackage{amsmath, amssymb}%数式
%\usepackage{array, booktabs}%表成型
%\usepackage[dvipdfmx]{graphicx}%画像

%\begin{document}

\subsection{解析手法B}
\label{subsec:PSAnalyses}
% コインシデンス,フィッティングは他の節はアルファベット表記が多い
\subsubsection{使用データ}
\label{subsubsec:PSData}
実験で得られたデータのうち,解析に用いたのは3 日目に磁場標的を用いたランと4 日目に銅板標的を用いたランとの2 つであり,それぞれのデータを表~\ref{tab:PSdata} に示す.
\begin{table}[h]
	\centering
	\caption{PS の解析に用いたデータ}
	\begin{tabular}{ccc} \toprule
	標的の種類 & 磁場$B~[\mathrm{G}]$ & Event 数 \\ \midrule
	銅板標的 & --- & 43502 \\
	磁場標的 & 53.97 & 448073 \\ \bottomrule
	\end{tabular}\label{tab:PSdata}
\end{table}%

なお,磁場標的を用いたデータについては先に述べたように磁場がランの途中で変化していたので,解析には変化後のデータとして,最初の30000~events を除いたデータのみを用いた.
詳しくは\ref{subsubsec:PSMagChangeCheck}で述べる.

\subsubsection{WFD波形の解析}
\label{subsubsec:PSEventDisplay}
WFD (CAEN Waveform Digitizer V1721) で得られたイベント / 波形の例を図~\ref{fig:PSEventDisplayAll}, \ref{fig:PSEventDisplayZoom} に示す.
\begin{figure}[h]
	\centering
	\includegraphics[width = 0.9\textwidth]{figure/odagawa/PSEventDisplayAll.png}
	\caption{プラスチックシンチレータで得られた波形 ($8~\mu\mathrm{s}$ 全体).線の種類がそれぞれのチャンネルに対応しており,層の番号は標的に近い方から1 から4 までである.}
	\label{fig:PSEventDisplayAll}
\end{figure}%
\begin{figure}[h]
	\centering
	\includegraphics[width = 0.9\textwidth]{figure/odagawa/PSEventDisplayZoom.png}
	\caption{プラスチックシンチレータで得られた波形 (一部拡大).図~\ref{fig:PSEventDisplayAll} の2.6 - 3.1~$\mu$s を拡大したものである.}
	\label{fig:PSEventDisplayZoom}
\end{figure}%
図~\ref{fig:PSEventDisplayZoom} からもわかる通り,プラスチックシンチレータのデータについては一つの信号が立ち下がる前に次の信号を検出してしまう「パイルアップ」はほとんどないものと考えられる.
また,立ち上がりが十分に早くない場合には,threshold を信号ごとにすべて同じにしたとき,threshold を超える時間が信号の高さで変化し,時刻がきちんと得られないことがある.
このような時刻のずれを補正するためにTQ 補正などが行われることがあるが,図~\ref{fig:PSEventDisplayAll} から信号の立ち上がりは十分早く,そのような補正は必要ないと判断した.
以降ではこれらを考慮したうえで,固定threshold を,プラスチックシンチレータのアフターパルスやベースラインの揺らぎを無視できる大きさ (8~WFD Count, 30~mV相当) に設定して解析を行った.

\subsubsection{ミューオン寿命解析}
\label{subsubsec:PSLife}
まず,銅板標的を用いたランのデータからミューオンの寿命を求めた.
\ref{subsubsec:PSEventDisplay} で述べたようにthreshold を設定し,固定threshold を越えたところから初めてthreshold 以下になるところまでを一つの崩壊$e^{+}$ による信号として,threshold を超えた瞬間をその信号が持つ時刻とした.
その後,3 層目を除く各層については以下の方法でコインシデンスをとった.

各層について,時間情報が$10~\mathrm{ns}$ 以内にあれば同じ崩壊$e^{+}$ 由来の信号として,二つの平均時間を各層の時間情報とした.
ここで10~ns という値は図~\ref{fig:PSEventDisplayZoom} などからプラスチックシンチレータの立ち下がり時間程度になるように選んだ.

以上のセレクション後に実際に得られた3 層目を除く各層の$e^{+}$ 検出時刻分布のヒストグラムを図~\ref{fig:PSLifeDist} に示す.
ただし2, 4 層目は1 層目とのコインシデンスをとった.
ここで3 層目を除いたのはチャンネル数の都合上,3 層目では層内でのコインシデンスをとることができなかったからである.
また,コインシデンスをとった方法は各層での方法と同じであるが,時間情報としては1 層目と各層との平均を用いるのではなく,コインシデンスが取れたものを各層についてそのまま使用した.
\begin{figure}[h]
	\centering
	\begin{minipage}{0.45\textwidth}
	\centering
	\includegraphics[width = \textwidth]{figure/odagawa/PSLifetimeDist_Layer0.png}
	\end{minipage}
	\begin{minipage}{0.45\textwidth}
	\centering
	\includegraphics[width = \textwidth]{figure/odagawa/PSLifetimeDist_Layer1.png}
	\end{minipage}
	\begin{minipage}{0.45\textwidth}
	\centering
	\includegraphics[width = \textwidth]{figure/odagawa/PSLifetimeDist_Layer3.png}
	\end{minipage}
	\begin{minipage}{0.45\textwidth}
	\centering
	\includegraphics[width = \textwidth]{figure/odagawa/PSLifetimeDistNoCoin_Layer3.png}
	\end{minipage}
	\caption{磁場がないときの時刻分布ヒストグラム.それぞれ (左上) 1 層目,(右上) 2 層目コインシデンスあり,(左下) 4 層目コインシデンスあり,(右下) 4 層目コインシデンスなしの時刻情報である.崩壊現象に特徴的な指数関数に従った減衰が見られる.}
	\label{fig:PSLifeDist}
\end{figure}%

得られた時間分布における$t = 200$ - $400~\mathrm{ns}$ のピークは,図~\ref{fig:PSLifeDist} の下二つを比較すると分かるように,コインシデンスをとることによって小さくなる.
このピークは$\pi$ の崩壊によってできた表面ミューオンがすぐに崩壊し生成した$e^{+}$ が,ビームラインを通り抜けて直接検出器に当たったときの信号によるものと考えられる.
ビームラインにおいては運動量で粒子を選択しており (今回は運動エネルギーが4.1~MeV の粒子を選択している) ,このようにしてできた$e^{+}$ は同じ運動量の$\mu^{+}$ よりも高速で検出器に到達する.
1 層目とのコインシデンスをとることでこのピークが低くなるのは,このコインシデンスにより粒子の飛来した方向を制限でき,銅板標的由来の信号の割合が多くなるためと考えられる.

図~\ref{fig:PSLifeDist_Layer0} - \ref{fig:PSLifeDist_Layer3} のヒストグラムに式\eqref{eq:PSLifeFitFunc} 
\begin{equation}
f(t) = A \exp(-t / \tau) + \mathrm{const.}
\label{eq:PSLifeFitFunc}
\end{equation}
で表される関数$f(t)$を用いてフィッティングを行った.
ここでバックグラウンドの影響を加味して定数項を加え,フィッテイング範囲は1000~ns から8000~ns とした.
フィッティング結果を図~\ref{fig:PSLifeFit} および表~\ref{tab:PSLifetime} に示す.誤差は統計によるものである.
\begin{figure}[h]
	\centering
	\begin{minipage}{0.45\textwidth}
	\centering
	\includegraphics[width = 0.45\textwidth]{figure/odagawa/PSLifetimeFit_Layer0.png}
	\end{minipage}
	\begin{minipage}{0.45\textwidth}
	\centering
	\includegraphics[width = \textwidth]{figure/odagawa/PSLifetimeFit_Layer1.png}
	\end{minipage}
	\begin{minipage}{0.45\textwidth}
	\centering
	\includegraphics[width = \textwidth]{figure/odagawa/PSLifetimeFit_Layer3.png}
	\end{minipage}
	\caption{図~\ref{fig:PSLifeDist} のヒストグラムに$f(t)$ でフィッティングを行った結果.それぞれ (上) 1 層目,(左下) 2 層目コインシデンスあり,(右下) 4 層目コインシデンスありのものである.}
	\label{fig:PSLifeFit}
\end{figure}%
\begin{table}[h]
	\centering
	\caption{フィッティングによって得られた$\tau$ の値}
	\begin{tabular}{cc}\toprule
	層番号 & $\tau~[\mathrm{ns}]$ \\ \midrule
	1 & $2247 \pm 16$ \\
	2 & $2264 \pm 23$ \\
	4 & $2150 \pm 46$ \\ \bottomrule
	\end{tabular}\label{tab:PSLifetime}
\end{table}%

ここで各層から得られた3~つの値を統計的にまとめてやると,ミューオンの寿命として$\tau = 2220 \pm 18~\mathrm{ns}$ という値が得られた.

\newpage

\subsubsection{ミューオン$g$ 因子解析}
\label{subsubsec:PSgFactor}
次に磁場標的を用いたランの解析から,ミューオンの$g$ 因子を求めた.
ここで先述の通り,磁場の変化が起こる前の最初の30000~events を除いたデータを用いた.
\ref{subsubsec:PSLife} と同様に得られた時刻情報を図~\ref{fig:PSgFactorDist} に示す.
\begin{figure}[h]
	\centering
	\begin{minipage}{0.45\textwidth}
	\centering
	\includegraphics[width = 0.45\textwidth]{figure/odagawa/PSgFactorDist_Layer0.png}
	\end{minipage}
	\begin{minipage}{0.45\textwidth}
	\centering
	\includegraphics[width = \textwidth]{figure/odagawa/PSgFactorDist_Layer1.png}
	\end{minipage}
	\begin{minipage}{0.45\textwidth}
	\centering
	\includegraphics[width = \textwidth]{figure/odagawa/PSgFactorDist_Layer3.png}
	\end{minipage}
	\caption{磁場があるときの時刻分布ヒストグラム.それぞれ (上) 1 層目,(左下) 2 層目コインシデンスあり,(右下) 4 層目コインシデンスありのものである.指数関数減衰にミューオンの歳差運動由来と思われる振動が重なっている.}
	\label{fig:PSgFactorDist}
\end{figure}%

図~\ref{fig:PSgFactorDist} のヒストグラムに式\eqref{eq:PSgFactorFitFunc}
\begin{equation}
g(t) = A \exp(-t / \tau) [1 + B \cos(\delta + \omega t)] + \mathrm{const.}
\label{eq:PSgFactorFitFunc}
\end{equation}
で表される関数$g(t)$ を用いてフィッティングを行った.
ここで\ref{subsubsec:PSLife} のときと同様に定数項を加え,フィッティング範囲を振動がきちんと見えている1000~ns から8000~ns までとした.
フィッティング結果を図~\ref{fig:PSgFactorFit},および表\ref{tab:PSgFactor} に示す.
ただし表~\ref{tab:PSgFactor} において$g$ 因子の計算には各点の磁場をビームプロファイルのガウシアンで加重平均した値,$B = 53.97~\mathrm{G}$ という値を用いた.
また,誤差としては\ref{subsubsec:PSLife} と同様にフィッティングの統計によるものを載せており,磁場の影響を含めたものについては後述する.
\begin{figure}[h]
	\centering
	\begin{minipage}{0.45\textwidth}
	\centering
	\includegraphics[width = 0.45\textwidth]{figure/odagawa/PSgFactorFit_Layer0.png}
	\end{minipage}
	\begin{minipage}{0.45\textwidth}
	\centering
	\includegraphics[width = \textwidth]{figure/odagawa/PSgFactorFit_Layer1.png}
	\end{minipage}
	\begin{minipage}{0.45\textwidth}
	\centering
	\includegraphics[width = \textwidth]{figure/odagawa/PSgFactorFit_Layer3.png}
	\end{minipage}
	\caption{図~\ref{fig:PSgFactorDist} のヒストグラムに$f(t)$ でフィッティングを行った結果.それぞれ (上) 1 層目,(左下) 2 層目コインシデンスあり,(右下) 4 層目コインシデンスありのものである}
	\label{fig:PSgFactorFit}
\end{figure}%
\begin{table}[h]
	\centering
	\caption{$g$ 因子フィッティング結果}
	\begin{tabular}{cccc} \toprule
	層番号 & $\tau~[\mathrm{ns}]$ & $\omega~[/ \mu\mathrm{s}]$ & $g$ \\ \midrule
	1 & $2249 \pm \phantom{0}4$ & $4.624 \pm 0.004$ & $2.013 \pm 0.0015$ \\  
	2 & $2224 \pm \phantom{0}6$ & $4.620 \pm 0.003$ & $2.012 \pm 0.0015$ \\  
	4 & $2066 \pm 13$ & $4.633 \pm 0.005$ & $2.017 \pm 0.0023$ \\  \bottomrule
	\end{tabular}\label{tab:PSgFactor}
\end{table}%

\newpage

\subsubsection{磁場の系統誤差について}
\label{subsubsec:MagSysErr}
\ref{subsubsec:PSgFactor} において$g$ 因子解析に用いた磁場の値$B = 53.97~[\mathrm{G}]$ は事前にMLF の三宅さんから頂いたビームプロファイルをもとに加重平均をとった値であるが,このビームプロファイルの不定性により加重平均の値は変化する.
今回はビームの広がりはガウシアンにしたがって分布すると仮定したうえでガウシアンの広がりを変化させ,それによる磁場への影響をみた.

ガウシアンの$x$ 方向の広がりを$\sigma_{x}$,$y$ 方向の広がりを$\sigma_{y}$ とし,$\sigma_{x}$ を2.0~cm から3.9 ~cm まで,$\sigma_{y}$ を1.0~cm から2.9~cm まで0.1~cm 刻みで変化させて加重平均磁場の値を求めた.得られた磁場の最大値・最小値を表~\ref{tab:MagSysErr} に示す.ここで用いた磁場の値はすべて表~10のものである.
ここで$\sigma_{x}, \sigma_{y}$ を動かす範囲は磁場標的の大きさが事前にいただいたビームプロファイルのガウシアンの標準偏差$1\sigma$ に入る点を上限として十分な広さをとった.
\begin{table}[h]
	\centering
	\caption{$\sigma_{x}$,$\sigma_{y}$を動かしたときの磁場$B$ の最大値と最小値~[G]}
	\begin{tabular}{cc}\toprule
	$B_{\mathrm{max}}$ & $B_{\mathrm{min}}$ \\ \midrule
	$54.44 (\sigma_{x} = 3.9, \;\sigma_{y} = 1.0)$ & $53.76 (\sigma_{x} = 3.9, \;\sigma_{y} = 2.9)$ \\ \bottomrule 	
	\end{tabular}\label{tab:MagSysErr}
\end{table}%

得られた磁場の最大値,および最小値を用いて計算した$g$ 因子の値を表~\ref{tab:PSgSysErr} に示す.
\begin{table}[h]
	\centering
	\caption{磁場$B$ の値とそれらに対応する$g$ 因子の値}
	\begin{tabular}{cccc}\toprule
	層番号 & $B_{\mathrm{max}} = 54.44~\mathrm{G}$ & $B_{0} = 53.97~\mathrm{G}$ & $B_{\mathrm{min}} = 53.76~\mathrm{G}$ \\ \midrule
	1 & 2.000 & 2.013 & 2.021 \\
	2 & 1.995 & 2.012 & 2.020 \\
	4 & 2.000 & 2.017 & 2.025 \\ \bottomrule 
	\end{tabular}\label{tab:PSgSysErr}
\end{table}%

また,磁場測定機器の測定誤差を含んだ$g$ 因子の統計誤差については,誤差伝播の式,式\eqref{eq:PSgosa} を用いて計算した.
$\delta B = 0.008~\mathrm{G}$となったので,$\delta g$ は表~\ref{tab:PSgStatErr} のようになった.

\begin{equation}
\delta g = \sqrt{\left(\delta\omega\right)^{2}\left(\frac{\partial g}{\partial\omega}\right)^{2} + \left(\delta B\right)^{2}\left(\frac{\partial g}{\partial B}\right)^{2}}
\label{eq:PSgosa}
\end{equation}%

\begin{table}
	\centering
	\caption{磁場測定機器の測定誤差を含んだ$g$ 因子の統計誤差}
	\begin{tabular}{cc}\toprule
	層番号 & $g$ 因子の統計誤差\\ \midrule
	1 & 0.008 \\ 
	2 & 0.008 \\
	4 & 0.009 \\ \bottomrule
	\end{tabular}\label{tab:PSgStatErr}
\end{table}%

実際,$g$ 因子の値を考えるときは1 層目と4 層目とのコインシデンスをとったものが銅板にミューオンがあたった上で検出器に飛来したような貫通イベントをとれている可能性が高く磁場の影響を受けた$e^{+}$ が多いと考えられることからも妥当であると考えると,得られた$g$ 因子の値は
\[g = 2.017 \pm 0.009 ^{+0.008}_{-0.017}\]
となった.
ここで誤差の第一項は統計誤差であり,第二項はビームプロフィルの不定性による系統誤差である.

%以下二つはAppendix にまわす?
%%%%%%%%%%%%%%%
\subsubsection{磁場の変化の確認}
\label{subsubsec:PSMagChangeCheck}
$g$ 因子測定のための磁場標的を置いたランの途中で磁石配置が崩れ,磁場の値が変わっていたことは本文中でも述べたが,解析においてそのことを確認した.

\ref{subsubsec:PSgFactor} のようなフィッティングをする際に,用いるイベントを9000~events (6~min) ごとに区切ってフィッティングを行った,それぞれ得られた$\omega$ をの値を図~\ref{fig:PSOmegaCheck} に示す.
赤線は最初の8~分間のデータから得られた値を表しており,また青線は磁場変化発覚後に改めて確認用にとったランから得られた値を表している.
ここで磁場変化発覚後のランはおよそ30~分間である.図~\ref{fig:PSOmegaCheck} を見ると分かるように,ランの最初の20~分程度で磁場の値が変化している.
よって今回の解析には同じ値の磁場とみなせるという条件の下で,より統計量の多いランの20~分以降のデータを用いることにした.
なお,変化前後の加重平均をとった磁場の値はそれぞれ56.06~G と53.97~G であった.
\begin{figure}[h]
	\centering
	\includegraphics[width = 0.9\textwidth]{figure/odagawa/PSOmegaCheck.png}
	\caption{ラン中における$\omega$ の時間変化}
	\label{fig:PSOmegaCheck}
\end{figure}

\subsubsection{NaI シンチレータとの位相差の確認}
\label{subsubsec:PhaseCheck}
NaI シンチレータとプラスチックシンチレータの$g$ 因子の解析から$\cos$ 振動の初期位相$\delta$ に注目すると,これら二つの位相差は実験中の物理的セットアップに起因していると考えられる.
実際,二つのシンチレータは標的に対しておよそ$87^{\circ}$ の角度をつけて配置していたため,それにともなって初期位相もおよそ$87^{\circ}$ 分の差をもっているはずである.

これらの事実を解析から確かめるために,まず二つの検出器から得られたデータの時間原点をそろえる必要がある.
これら二つの検出器間には時間分解能などの性能差もあるが,そもそも信号の伝送に用いたケーブルの長さが違うために時間原点がそろっていない.
今回は時間原点をそろえるために\ref{subsubsec:PSLife} でものべた高速$e^{+}$ 粒子による小さなピークを用いることを考えた.
今回のセットアップにおいて二つの検出器からビーム出口までの長さはビームライン全体の長さに比べると小さく,生成ターゲットから二つの検出器までに$e^{+}$ が飛来するまでの時間はほぼ等しいと考えられるので,このピークをそろえることで時間原点とすることができる.
NaI シンチレータの$e^{+}$ ピーク探索とそれを用いた原点調整後のヒストグラムを図~\ref{fig:NaITimeOrigin}に示す.また,プラスチックシンチレータで同様のことを行ったものを図~\ref{fig:PSTimeOrigin} に示す.
\begin{figure}[h]
	\centering
	\includegraphics[width = 0.9\textwidth]{figure/odagawa/NaITimeOrigin.png}
	\caption{(左) NaI シンチレータにおける$e^{+}$ イベントの時刻分布.(右) 原点調節後のヒストグラム}
	\label{fig:NaITimeOrigin}
	\includegraphics[width = 0.9\textwidth]{figure/odagawa/PSTimeOrigin.png}
	\caption{(左) プラスチックシンチレータにおける$e^{+}$ イベントの時刻分布.(右) 原点調節後のヒストグラム}
	\label{fig:PSTimeOrigin}
\end{figure}

それぞれのフィッティングから得られた初期位相$\delta$ を表\ref{tab:InitPhases} に示す.
\begin{table}[h]
	\centering
	\caption{各検出器の初期位相}
	\begin{tabular}{cc}\toprule
	検出器 & $\delta~[\mathrm{rad.}]$ \\ \midrule
	NaI シンチレータ & $-0.54 \pm 0.04$ \\
	プラスチックシンチレータ & $\phantom{-}1.10 \pm 0.02$ \\ \bottomrule
	\end{tabular}\label{tab:InitPhases}
\end{table}%
表\ref{tab:InitPhases} から位相差は$1.64 \pm 0.06 \sim \pi / 2$ となり,セットアップの角度と整合していることが確認できた.

\subsection{PS の解析まとめ}
\label{subsec:PSmatome}
二つの解析手法を用いてミューオンの寿命,および$g$ 因子を求めた.これらの結果を表~\ref{tab:matome_PS} にまとめた.
\begin{table}[h]
	\centering
	\caption{二つの手法によるプラスチックシンチレータでの解析結果}
	\begin{tabular}{ccc}\toprule
	{} & 解析手法A & 解析手法B \\ \midrule
	$\tau$~[ns]& $2223 \pm 6_{- 75}$ & 2220 $\pm$ 18\\
	$g$ & $2.004 \pm 0.003^{+0.017}$  & $2.017 \pm 0.009 ^{+0.008}_{-0.017}$\\ \bottomrule
	\end{tabular}\label{tab:matome_PS}
\end{table}%

%二つの解析結果をまとめると,表~\ref{tab:matome_PS2} のようになった.ここでそれぞれの系統誤差はそのまま含めた.

%\begin{table}[H]
%	\centering
%	\caption{二つの解析結果をまとめた値}
%	\begin{tabular}{cc}\toprule
%	$\tau$~[ns]& $2222 \pm 9_{-75}$\\
%	$g$ & $2.010 \pm 0.005 \pm 0.017$\\ \bottomrule
%	\end{tabular}\label{tab:matome_PS2}
%\end{table}%

%\end{document}

%\usepackage{here}
%\usepackage[dvipdfmx]{graphicx}
%\usepackage{amsmath,amssymb}
%\usepackage{array,booktabs} %表のためのarray環境

%\begin{document}
\section{NaIで取得したデータの解析と結果・考察}
以下では主に測定した信号から欲しい情報を取り出すための解析に関する部分と,得られた情報から実際の測定した量を解析する部分にわけて,解析の詳細を述べる.
特に前半の信号解析ではパイルアップ信号を取り除く処理を行い,情報を取り出す解析を行った.
後半では寿命と$g$ 因子の測定をフィッティングを用いて行った.
また,ミッシェルパラメータの測定では時間情報に基づいたスピンの情報とエネルギーの相関を含めた解析を行った.
その際,各NaIで得られたエネルギー情報に対し測定器の電磁シャワーの応答に基づくフィッティングを行った.

\subsection{信号解析}
NaI 検出器で測定した波形解析の手法について述べる.
測定波形の中で典型的なものを\figref{hatano_fig:rawdata} に示す.
波形解析ではここからフィンガーカウンター(チャンネル1 )が鳴っているという条件を課したときにNaI で測定したエネルギーと信号の時刻という2 つの情報を抽出した.
この際に信号のノイズとNaI と波形の重なり(以下,パイルアップとよぶ)の2 つが解析時の注意点となった.
前者は信号が鳴ったという判定をするためのピーク検出をする際に問題となる.
具体的には,本来1 つの$e^+$ からの信号であるのにノイズの影響で複数のピークとして認識してしまう.
そのため,今回の解析ではノイズ除去の処理を最初に行った.
その後ピーク検出を行った後に時間情報とエネルギー情報を取り出す.
この際にパイルアップの処理をしないまま適当な領域で積分を行いエネルギーを求めると,別の$e^+$ からの信号でエネルギーが大きく見積もられてしまう.
このため,今回の解析では波形データのサンプリングを行い,それに基づきパイルアップしてない部分の波形からパイルアップしている部分に外挿し,その分の補正を行った.
以下具体的な処理について述べる.

\begin{figure}[hbt]
\centering
\includegraphics[width=0.6\textwidth]{figure/hatano/rawdata_modify1.eps}
\caption{NaI 検出器の測定信号.チャンネル1 はフィンガーカウンターの測定波形,チャンネル3 〜チャンネル7 はNaI の測定波形.NaIの測定波形では波形の重なりが見られる.}
\label{hatano_fig:rawdata}
\end{figure}

\subsubsection{ノイズ除去}
ピーク検出の際に問題となる高周波のノイズを除去することについて考える.
高周波成分によるピークと誤認識するような大きな変動がなくなればいいので,簡単に実装ができかつ高速な処理ができるという観点で単純移動平均をとるということを行った.
具体的には各サンプリング点で前後4 サンプリング,計9 サンプリングの平均をとった.
実際にノイズ除去を行った前後の信号の差異は\figref{hatano_fig:smoothdata}(\figref{hatano_fig:smoothdata} と同じデータの一部)のようになった.
チャンネル6 の左側の信号はノイズによって2 つのピークのように見かけ上分裂していたのが改善しているのが分かる.

\begin{figure}[hbt]
\centering
\includegraphics[width=0.6\textwidth]{figure/hatano/smoothdata_modify1.eps}
\caption{ノイズ除去された測定信号.薄い色で示したのが元の信号,濃い色で示したのがノイズ除去を行った信号.チャンネル6 の信号はノイズによって2 つのピークのように見かけ上分裂していたのが改善している.}
\label{hatano_fig:smoothdata}
\end{figure}

\subsubsection{ピーク検出}
以上のノイズ除去を行った信号を元にピーク検出を行うことを考える.
基本的にはthresholdを適切に設定しそれを超えたところからピークが始まったと考えることにする.
ただし,この方法ではパイルアップしている際には過剰に信号を検出してしまう可能性がある.
すなわち,別の信号と重なってそれがオフセットになりthreshold が下がっているのと同じ状態になり,低エネルギーの信号を拾いやすくなる.
そのため,threshold のbaseline となる値を信号が来る前の最小値とすることにした.その手法の模式図を\figref{hatano_fig:threshold}に示す.

\begin{figure}[hbt]
\centering
\includegraphics[width=0.6\textwidth]{figure/hatano/threshold.eps}
\caption{ピーク検出の模式図.青線と緑線で示したのが入射した粒子毎に対応する信号.赤線で示したのが青線と緑線の信号を合わせた測定信号.薄い青で示した矢印がthreshold の高さである.緑線に対応する信号はthreshold 以下であり検出されない.}
\label{hatano_fig:threshold}
\end{figure}

このような工夫をすることで,パイルアップしている際はbaseline が高くなるため,パイルアップによるオフセットの影響をある程度キャンセルすることができる.
そのようにしてピーク検出を行った際の信号が\figref{hatano_fig:peakdata} である.

\begin{figure}[hbt]
\centering
\includegraphics[width=0.6\textwidth]{figure/hatano/peakdata_modify1.eps}
\caption{ピークとして検出された信号.それぞれ色が濃くなっている部分がピークまでの立ち上がりとして判定されている領域で,点で示されているのがピーク(最大値)である.}
\label{hatano_fig:peakdata}
\end{figure}

\subsubsection{波形データの抽出}
パイルアップに対処するために全イベントからパイルアップしていない時の波形データを抽出しそれにより外挿を行うことを考える.
NaIは減衰時間が長く主にその部分でパイルアップが起きていると考えた.
すなわち,ピークの部分は全て検出できているとし立ち下がる部分によるパイルアップのみに対処した.
具体的には波形データのサンプルから減衰時間を測定し,それを元にピークから立ち下がり部分に$\exp(-t/\tau)$ ($\tau$: 減衰時間)の関数を仮定して外挿をするというアプローチをとった.
まず,ピーク検出された信号のうちパイルアップしていない信号の減衰部分に対して指数関数でフィッティングを行った.
パイルアップしていない信号は,次のピーク信号がくるまで400~ns 以上の間隔があるものと定義した.
また減衰部分の領域はピーク部分の影響やバックグラウンド部分の影響を減らすため,ピークから80~ns 後から次の信号が来るまでの最後の10~\% を外した領域とした.
これに対してフィッティングを行いその$\chi^2$ のConfidence Level (CL) が50~\% を切るものは解析には使用しなかった.
実際のフィッティングを行った時の様子が\figref{hatano_fig:decayfit} である.濃い色で示されているのがフィッティングした関数であり,赤の線で示したのはCL が50~\% 以下であったため用いなかった信号である.

\begin{figure}[hbt]
\centering
\includegraphics[width=0.6\textwidth]{figure/hatano/decayfit_modify1.eps}
\caption{波形データに対するフィッティング.濃い色で示されているのがフィッティングした関数であり,赤の線で示したのはCL が50~\% 以下であったため用いなかった信号である.}
\label{hatano_fig:decayfit}
\end{figure}

これを全ての測定データに行い,減衰時間の分布を求めた結果が\figref{hatano_fig:decaytime} である.
チャンネル7 のヒストグラムは2 峰になっているが,これは異なる減衰特性の2 つのNaI の信号をアナログ的に加算したためと考えられる.
他のチャンネルのデータのピーク値もある程度のばらつきがある事が確認される.
また外挿の際にはピークからそれがアナログ的に加算したNaI のどちらからの信号であるかを識別する事はできないため,これらを選別することを考えず各チャンネル毎に減衰時間を単純に平均し,それを減衰時間とした.

\begin{figure}[hbt]
\centering
\includegraphics[width=0.6\textwidth]{figure/hatano/decaytime.eps}
\caption{NaI の減衰時間の測定.チャンネル7が2峰あるのは異なる減衰特性の2つのNaIの信号をアナログ的に加算したためと考えられる.}
\label{hatano_fig:decaytime}
\end{figure}

その結果が\tabref{hatano_tab:decaytime}である.

\begin{table}[hbt]
\centering
\caption{各チャンネルにおける減衰時間}
\begin{tabular}{cc}\\ \toprule
チャンネル番号 & 減衰時間~[ns]\\ \midrule
3 & 232.6 \\
4 & 236.7 \\
5 & 224.2 \\
6 & 233.4 \\
7 & 228.6 \\ \bottomrule
\end{tabular}
\label{hatano_tab:decaytime}
\end{table}

\subsubsection{波形データの外挿}
波形データの外挿は次のように行った.
ピークが終わった後の減衰中に次のピークが見つかった場合は,減衰時間中の最小値から前5サンプリング分のデータを基準に求めた減衰時間(\tabref{hatano_tab:decaytime}) の指数関数を外挿した.
実際に外挿を行った時の信号の様子が\figref{hatano_fig:analysis} であり,濃い色の線で示しているのが外挿した波形データである.

\begin{figure}[hbt]
\centering
\includegraphics[width=0.6\textwidth]{figure/hatano/analysis_modify1.eps}
\caption{波形の外挿の様子.濃い色の線で示しているのが外挿された波形データ.点で示しているのが各信号の時間と定義されたところ.}
\label{hatano_fig:analysis}
\end{figure}

\subsubsection{解析データの抽出}
以上で各チャンネルごとに信号から時刻とエネルギーの情報を抽出する準備ができたので,これを元にピークの50~\%の値を超えたところを信号の時刻として,前のピークから外挿された分を差し引き,自らの外挿分を加えた積分値をエネルギーとした.

必要なデータは標的から飛んできた陽電子が手前のフィンガーカウンターを通った事象のものだけであるので,フィンガーカウンターの信号と同時に鳴ったとみなせるNaI 信号を選ぶことを考える.
フィンガーカウンターと各NaI信号の時間差をとると\figref{hatano_fig:coincidence}のようになった.
この分布より,フィンガーカウンターが鳴った時を基準にその後20~nsの間に鳴ったNaI 信号をフィンガーカウンターと同時に鳴ったものと定義した.

\begin{figure}[hbt]
\centering
\includegraphics[width=0.6\textwidth]{figure/hatano/coincidence.eps}
\caption{フィンガーカウンターと各NaI の時間差}
\label{hatano_fig:coincidence}
\end{figure}

%%%%%%%%ここから三野の続き%%%%%%

\subsection{NaI を用いた寿命と$g$ 因子の解析}
\subsubsection{使用データ}
NaI では実験で得られたデータのうち,寿命測定用に磁場なしのデータとしてRUN15,$g$ 因子測定用に磁場ありのデータとしてRUN18, RUN19 のデータを用いて解析を行った.
RUN15 のサンプル数は1030点でビームが到達してから4120~ns,RUN18 とRUN19 のサンプル数は2050 点で8200~ns の時間までのデータを記録した.
解析に用いたRUNは表\ref{tab:RUN_info} のようなデータであった.

\begin{table}[H]%RUN Information
\caption{用いたRUN の情報}
\centering
\begin{tabular}{cccc}\toprule
{} & $B~[\mathrm{G}]$ & Time~[min] & Event数\\ \midrule
RUN15 & --- & 47 & 71532 \\
RUN18 & 56.06 & 75 & 113584 \\
RUN19 & 53.97 & 297 & 446578 \\ \bottomrule
\end{tabular}
\label{tab:RUN_info}
\end{table}

図\ref{fig:no_mag}-\ref{fig:with_mag_RUN19} は解析に用いた磁場なしと磁場ありの場合の時間分布で中心のNaI とフィンガーカウンターでコインシデンスを取った.
% コインシデンスという表現でいいか?
\begin{figure}[H]
\centering
\includegraphics[width  = 0.5\textwidth]{figure/mino/no_mag.png}
\caption{磁場がないときの時間分布 (RUN15)}
\label{fig:no_mag}
\begin{minipage}{0.45\hsize}
\centering
\includegraphics[width  = 1.0\textwidth]{figure/mino/with_mag_RUN18.png}
\caption{磁場があるときの時間分布 (RUN18)}
\end{minipage}
\begin{minipage}{0.45\hsize}
\centering
\includegraphics[width  = 1.0\textwidth]{figure/mino/with_mag_RUN19.png}
\caption{磁場があるときの時間分布 (RUN19)}
\label{fig:with_mag_RUN19}
\end{minipage}
\end{figure}

%--------- peak ratio -----------------------------------------------------------

\subsubsection{時間分解能}
ピークに対する一定の高さの比 (50~\%) を超えたところを信号の時間として用いて寿命と$g$ 因子のフィッティングを行った.図\ref{fig:Original} は中心のNaI とフィンガーカウンターでコインシデンスを取った時の図で, 縦軸が時間差,横軸が中心のNaI で落としたエネルギーを表しており,この図からエネルギーに対して時間差がほとんど一定でTQ補正の必要はないことを確認した.

\begin{figure}[H]%TQ compensation check
\centering
\includegraphics[width  = 0.5\textwidth]{figure/mino/Original.png}
\caption{中心のNaI とフィンガーカウンターの時間差}
\label{fig:Original}
\end{figure}

ただし,時間分解能が低エネルギーでは波形信号が小さい影響で高エネルギー側と比べて悪いため,5~MeV より高いエネルギーの信号を用いて解析を行った.図~\ref{fig:NaI_peak_gaus_fitting} は1~MeV 毎にエネルギーで区切って時間をガウシアンでフィッティングしたものである.また,図~\ref{fig:NaI_peak_time_resolution} はそうして得られた時間分解能$\sigma$ を横軸をエネルギーとしてプロットしたもので低エネルギーは時間分解能が悪いことが確認できる.

\begin{figure}[H]%time resolution
\begin{minipage}{0.5\hsize}
\centering
\includegraphics[width  = 0.8\textwidth]{figure/mino/gausfitting_ratio.png}
\caption{ガウシアンのフィッティング}
\label{fig:NaI_peak_gaus_fitting}
\end{minipage}
\begin{minipage}{0.5\hsize}
\centering
\includegraphics[width  = 0.8\textwidth]{figure/mino/timeresolution_ratio.png}
\caption{1~MeV 毎に時間分解能をプロット}
\label{fig:NaI_peak_time_resolution}
\end{minipage}
\end{figure}

%-------- lifetime fitting ----------------------------------------------------------

\subsubsection{ミューオン寿命解析}
銅板標的を用いたRUN15 の解析からミューオンの寿命を求めた.
得られた時間分布に対して以下の式~\eqref{eq:lifetime} で表される寿命の関数$f(t)$を用いてフィッティングを行った.バックグラウンドの影響を加味して定数項を加え,RUN15 のデータは4120~ns までしかデータを記録していなかったため,フィッティング範囲は1200~ns から4000~ns とした.統計誤差はROOTのフィッティングによるものである.
%プラスチックシンチレータ側は定数項を含まないフィッティングもしている。
\begin{gather}
f(t) = A\exp(-t / \tau)+C \label{eq:lifetime}\\
\tau = 2.184 \pm 0.052~[\mu \mathrm{s}] \notag
\end{gather}
\begin{figure}[H]
\centering
\includegraphics[width  = 0.7\textwidth]{figure/mino/lifetime_NaI_ratio.png}
\caption{寿命フィッティング}
\end{figure}

%--------- g factor ---------------------------------------------------------------

\subsubsection{ミューオン$g$ 因子解析}

次に磁場標的を用いたRUN18 とRUN19 の解析からミューオンの$g$ 因子を求めた.RUN18 のデータは全Event を用いて解析を行ったが,RUN19 は先述の通り途中で磁石が外れて磁場の値が変化したため,磁場の変化を確認した結果から最初の30000~Events を除いたデータを用いた.

RUN18 とRUN19 から得られた時間分布に対して式\eqref{eq:gfactor} で表される$g$ 因子の関数$g(t)$を用いてフィッティングを行った.ここで寿命解析の時と同様に定数項を加え,フィッティング範囲は振動がきちんと見えている1600~ns から8000~ns までとした.

RUN18 とRUN19 のフィッティング結果をまとめると表~\ref{tab:gfactor_result} となった.
ただし表~\ref{tab:gfactor_result} において$g$ 因子の計算にはプラスチックシンチレータの場合と同様に各点の磁場をビームプロファイルのガウシアンで加重平均した値,RUN18では$B$=56.06~G,RUN19では$B$=53.97~G を用いた.
また,統計誤差としてはROOTのフィッティングによって得られたものを載せており,磁場の影響を含めたものについては後述する.

\begin{gather}
g(t) = A\exp(-t / \tau)\{1+B\cos(\omega t+\delta)\}+C\label{eq:gfactor}
\end{gather}
\begin{table}[H]
\caption{フィッティングで得られたRUN18 とRUN19 の$\omega$ と$g$ 因子}
\centering
\begin{tabular}{cccc}\toprule
{} & $\tau~[\mu \mathrm{s}]$ & $\omega~[/\mu \mathrm{s}]$ & $g$ \\ \midrule
RUN18 & 2.010 $\pm$ 0.048 & 4.923 $\pm$ 0.024 & 2.086 $\pm$ 0.010  \\
RUN19 & 2.126 $\pm$ 0.030 & 4.630 $\pm$ 0.015 & 2.038 $\pm$ 0.007 \\ \bottomrule
\end{tabular}
\label{tab:gfactor_result}
\end{table}

\begin{figure}[H]
\begin{minipage}{0.5\hsize}
\includegraphics[width  = 1.0\textwidth]{figure/mino/gfactor_ratio_RUN18.png}
\caption{RUN18 を用いた$g$ 因子のフィッティング}
\end{minipage}
\begin{minipage}{0.5\hsize}
\includegraphics[width  = 1.0\textwidth]{figure/mino/gfactor_ratio_RUN19.png}
\caption{RUN19 を用いた$g$ 因子のフィッティング}
\end{minipage}
\end{figure}

%-------- systematic error -------------------------------------------------------

\subsubsection{磁場の系統誤差について}

プラスチックシンチレータの場合と同様にRUN18 とRUN19 の$g$ 因子解析に用いた磁場の値はビームプロファイルをもとに加重平均をとった値であるため,ビームの不定性による磁場の
加重平均の誤差を考察した.\\
ビーム強度のガウシアンの広がり$\sigma_{x},\sigma_{y}$ を動かしてRUN18 とRUN19 の場合での加重平均磁場の最大値と最小値を求めたところ表~\ref{tab:mag_max_min} のようになった.

\begin{table}[H]
\caption{$\sigma_{x},\sigma_{y}~[\mathrm{cm}]$ を動かした時の最大磁場と最小磁場}
\centering
\begin{tabular}{ccc}\toprule%最大磁場と最低磁場
{} & $B_\mathrm{max}~[\mathrm{G}]$ &  $B_\mathrm{min}~[\mathrm{G}]$   \\ \midrule
RUN18 & 56.58\;($\sigma_{x}=3.3,\sigma_{y}=1.0$) & 55.82\;($\sigma_{x}=3.9,\sigma_{y}=2.9$)  \\
RUN19 & 54.44\;($\sigma_{x}=3.9,\sigma_{y}=1.0$) & 53.76\;($\sigma_{x}=3.9,\sigma_{y}=1.0$) \\ \bottomrule
\end{tabular}
\label{tab:mag_max_min}
\end{table}

磁場が最大・最小,そしてもとのビームプロファイルのときの$g$ 因子を求めると表\ref{tab:mag_g}の値が得られた.また,磁場の測定誤差を含んだ$g$ 因子の統計誤差については,磁場の誤差$\delta B$ を考慮するとプラスチックシンチレータで求めた誤差伝播の式\eqref{eq:PSgosa}より,$g$ 因子の誤差$\sigma_{B}$ は表~\ref{tab:g_error} の値が得られた.

\begin{table}[H]
\caption{磁場$B$ の値とそれらに対応する$g$ 因子の値}
\centering
\begin{tabular}{cccc}\toprule%最大磁場と最低磁場
{} & $B_\mathrm{max}$ & $B$ & $B_\mathrm{min}$  \\ \midrule
RUN18 & 2.066 & 2.086 & 2.095 \\
RUN19 & 2.020 & 2.038 & 2.046 \\ \bottomrule
\end{tabular}
\label{tab:mag_g}
\end{table}

\begin{table}[H]%g因子の系統誤差(磁場)
\caption{磁場による$g$ 因子の誤差の伝播}
\centering
\begin{tabular}{ccc}\toprule
{} & $\delta B~[\mathrm{G}]$ &  $\sigma_{B}$  \\ \midrule
RUN18 & 1.20 & 0.046  \\
RUN19 & 2.36 & 0.089 \\ \bottomrule
\end{tabular}
\label{tab:g_error}
\end{table}

以上の統計誤差と系統誤差をまとめると,RUN18 とRUN19 の$g$ 因子は表\ref{tab:NaIggosamatome}のようになった.
ここで誤差の第一項は磁場の影響を含めた統計誤差であり,第二項はビームプロファイルの不定性による系統誤差である.

\begin{table}[H]%g因子の誤差のまとめ
\caption{$g$ 因子の誤差のまとめ}
\centering
\begingroup
\renewcommand{\arraystretch}{1.2}%行間を変更
\begin{tabular}{cc}\toprule
{} &   $g$  \\ \midrule
RUN18 & $2.086 \pm 0.046^{+0.009}_{-0.020} $  \\
RUN19 & $2.038 \pm 0.089^{+0.008}_{-0.018} $  \\ \bottomrule
\end{tabular}\label{tab:NaIggosamatome}
\endgroup
\end{table}

したがって,二つの値をまとめて,$g = 2.062 \pm 0.050^{+0.009}_{-0.020}$ という値が得られた.ここで系統誤差は二つのうち大きい方を選んだ.

\subsubsection{プラスチックシンチレータとの位相差の確認}
\label{subsubsec:PhaseCheck}

NaI シンチレータとプラスチックシンチレータの$g$ 因子の解析から$\cos$ 振動の初期位相$\delta$ に注目すると,これら二つの位相差は実験中の物理的セットアップに起因していると考えられる.
実際,二つのシンチレータは標的に対しておよそ$87^{\circ}$ の角度をつけて配置していたため,それにともなって初期位相もおよそ$87^{\circ}$ 分の差をもっているはずである.

これらの事実を解析から確かめるために,まず二つの検出器から得られたデータの時間原点をそろえる必要がある.
これら二つの検出器間には時間分解能などの性能差もあるが,そもそも信号の伝送に用いたケーブルの長さが違うために時間原点がそろっていない.
今回は時間原点をそろえるために\ref{subsubsec:PSLife} でものべた高速$e^{+}$ 粒子による小さなピークを用いることを考えた.
今回のセットアップにおいて二つの検出器からビーム出口までの長さはビームライン全体の長さに比べると小さく,生成ターゲットから二つの検出器までに$e^{+}$ が飛来するまでの時間はほぼ等しいと考えられるので,このピークをそろえることで時間原点とすることができる.
NaI シンチレータの$e^{+}$ ピーク探索とそれを用いた原点調整後のヒストグラムを図~\ref{fig:NaITimeOrigin}に示す.また,プラスチックシンチレータで同様のことを行ったものを図~\ref{fig:PSTimeOrigin} に示す.
\begin{figure}[h]
	\centering
	\includegraphics[width = 0.9\textwidth]{figure/mino/NaITimeOrigin.png}
	\caption{(左) NaI シンチレータにおける$e^{+}$ イベントの時刻分布.(右) 原点調節後のヒストグラム}
	\label{fig:NaITimeOrigin}
	%\includegraphics[width = 0.9\textwidth]{figure/mino/PSTimeOrigin.png}
	\includegraphics[width = 0.9\textwidth]{figure/mino/PSOriginCheck.eps}
	\caption{(左) プラスチックシンチレータにおける$e^{+}$ イベントの時刻分布.(右) 原点調節後のヒストグラム}
	\label{fig:PSTimeOrigin}
\end{figure}

それぞれのフィッティングから得られた初期位相$\delta$ を表\ref{tab:InitPhases} に示す.
\begin{table}[h]
	\centering
	\caption{各検出器の初期位相}
	\begin{tabular}{cc}\toprule
	検出器 & $\delta~[\mathrm{rad.}]$ \\ \midrule
	NaI シンチレータ & $-0.54 \pm 0.04$ \\
	プラスチックシンチレータ & $\phantom{-}1.11 \pm 0.02$ \\ \bottomrule
	\end{tabular}\label{tab:InitPhases}
\end{table}%
表\ref{tab:InitPhases} から位相差は$1.65 \pm 0.06 \sim \pi / 2$ となり,セットアップの角度と整合していることが確認できた.

%\end{document}

%\documentclass{jarticle}
%\newcommand{\figref}[1]{\figurename\ref{#1}}
%\newcommand{\tabref}[1]{\tablename\ref{#1}}
%\usepackage[dvipdfmx]{graphicx}
%\begin{document}

%\section{解析}
\subsection{エネルギー解析}
信号解析で求めたエネルギーの値を元にミッシェルパラメータを求めるための各種解析を行った.
まずは時間情報を元にスピンの向きに関する考察をし,さらにバックグラウンドの影響を考えてイベントのセレクションを行った.
また,エネルギースペクトルに対するフィッティングでは検出器内での電磁シャワーによるエネルギー応答を考慮して行列を関数に畳み込んで最小二乗法を適用した.

\subsubsection{スピンの回転に関して}
当初の計画とは異なるが$g$ 因子測定用の磁場標的データを用いて以下の解析を行った.
このデータを元に解析を行ってもミッシェルパラメータの測定ができることがわかったため,長い時間の測定データを使う方が統計誤差の観点から良いためである.
また,このデータはスピン方向の情報を持っているため,スピンに関わるミッシェルパラメータである$\xi,\delta$ の測定も可能である.
以下ではまず時間情報を元にスピン方向が求められる事をみる.

崩壊時間の分布は$g$ 因子の解析で説明したように減衰部分と振動部分の積の形で出てくる.
$\rho$ を求めるためには無偏極のデータを得る必要があるが,スピン歳差運動の一周期分の時間範囲のデータを取り出しても,指数の減衰があるためスピン方向を等価に足しあわせることができない.
そこで,減衰の逆数で重みをつけることを考える.
ここではミューオンの崩壊寿命$\tau=2.2~\mathrm{\mu s}$ を既知として各イベントについて$\exp(t/\tau)$の重み付けを行った.
すると崩壊時間重みつき分布は\figref{hatano_fig:oscillation} のようになった.
これは実際に指数関数の影響をキャンセルしてスピンに由来する情報のみを取り出せてることを示している.

\begin{figure}[hbt]
\centering
\includegraphics[width=0.6\textwidth]{figure/hatano/oscillation_modify1.eps}
\caption{時間に対するスピンによる計数の変動の様子.指数関数の逆数で重みをつけた.}
\label{hatano_fig:oscillation}
\end{figure}

以上より,同様の重み付けをした上で適切な時間範囲をとることで,任意のスピンの向きに関するエネルギースペクトルを取り出すことができる.

\subsubsection{イベントセレクション}
標的方向からの粒子はフィンガーカウンターが鳴っているという条件を課すと中心のNaI に大部分のエネルギーを落とすと考えられるので,バックグラウンドを減らすために中心のNaI と全体とのエネルギー比$\alpha$ を用いてイベントセレクションを行った.
比の値$\alpha$ について,$\alpha=0,0.2,0.4,0.6,0.8,1.0$ 以上のイベントを選んだ時のエネルギースペクトラムを\figref{hatano_fig:bg} に示す.
$\alpha=0.2,0.4,0.6,0.8$ ではイベント数が減るだけで分布の相対的な形は大きな変化をせず,以降の解析の値に大きな変化はなかった.
以下の解析では,$0.2\leq\alpha\leq0.8$ の範囲でミッシェルパラメータ$\rho$ のフィッティングの$\chi^2$ が最小となった$\alpha=0.6$ とした.
\begin{figure}[hbt]
\centering
\includegraphics[width=0.6\textwidth]{figure/hatano/bg_modify1.eps}
\caption{NaI で測定した$e^+$ のエネルギー分布.中心のNaI と全体のNaI とのエネルギー比を用いてカットを行ったときの変動を示す.}
\label{hatano_fig:bg}
\end{figure}

\subsubsection{検出器の電磁シャワー応答について}
今回の検出器はNaI結晶による全吸収型のカロリメータではあるが,検出器内で形成された電磁シャワーで主に$\gamma$ 線となったものが検出器内から漏れる事が多いため,必ずしも入射した粒子のエネルギーに比例したエネルギーが測定されるわけではなく,低エネルギーの裾をもつ分布になる.
ただし,この関数形を解析的な形で求めることは難しいので今回はGeant4 を用いたシミュレーションを行って,応答を計算した.
結果を\figref{hatano_fig:response}に示す.
観測するエネルギーは入射した$e^+$ のエネルギー付近にピークを持ち低ネルギーの裾を持つ分布になっていることが分かる.
観測するエネルギーが入射したエネルギーよりも高くなるのは,検出器中の$e^-$ との対消滅によるものと考えられる.

\begin{figure}[hbt]
\centering
\includegraphics[width=0.6\textwidth]{figure/hatano/response.eps}
\caption{検出器内の電磁シャワー応答のシミュレーション.縦軸は検出器に入射した$e^+$ のエネルギー,横軸は検出器で観測した$e^+$ のエネルギーを示す.}
\label{hatano_fig:response}
\end{figure}

以下の解析ではこの応答を畳み込んだ最小二乗法を用いた.
最小二乗法の詳細については付録Bに掲載する.

\subsubsection{ミッシェルパラメータ$\rho$ の導出}
スピンが無偏極のときのエネルギースペクトルは\eqref{hatano_eq:rho}と表される.
ここでは,無偏極データを得るために\figref{hatano_fig:oscillation}の最初の三周期に相当する範囲を抽出した.
この際,各イベントには指数関数の逆数で重みをつけた.
\begin{equation}
  f(x)=(3 - 3x / E_\mathrm{max})+\frac{2}{3}\rho(4x / E_\mathrm{max} - 3)
  \label{hatano_eq:rho}
\end{equation}
このデータに対して,\eqref{hatano_eq:rho}の関数でフィッティングを行うわけだが,このフィッティングは高エネルギー部分の分布に敏感であり,そのため較正係数の誤差の影響が大きい.
較正計数を測定誤差である$\pm 20~\%$ の範囲で動かし,$\chi^2$ が最小になったときを測定値とした.

フィッティングの結果を\figref{hatano_fig:rho} に示す.(バックグラウンドとして一次関数を仮定した).
この時のフィッティングのパラメータは$f(x)$を$p_0(3 - 3x / E_\mathrm{max}) + \frac{2}{3} p_{1} (4x / E_\mathrm{max} - 3)$ ,バックグラウンドを$p_2+p_3x$ と書くと,\tabref{hatano_tab:rho} に示す値になった.
これより,ミッシェルパラメータ$\rho$は$\rho=0.662 \pm 0.022$ と求まった.

\begin{table}[hbt]
\centering
\caption{$\rho$ のフィッティングパラメータ}
\begin{tabular}{cc|cc|cc|cc}
$p_0$ & $\delta p_0$ & $p_1$ & $\delta p_1$ & $p_2$ & $\delta p_2$ & $p_3$ & $\delta p_3$ \\ \hline
3.092 & 0.011 & 4.665 & 0.128 & 1476.440 & 0.158 & -2.660 & 0.011
\end{tabular}
\label{hatano_tab:rho}
\end{table}

最小二乗法ではフィッティングパラメータの推定値から$1\sigma$ ずれた時に$\chi^2$ の値は$chi^2_{min}+1$ となる.\cite{leo} 
較正からの系統誤差を求めるために,較正較正を動かし$\chi^{2}_{min} + 1$ 以下であるところを求めた.
最大で0.801,最小で0.617 であり$\rho=0.662$ との大きい方の差である0.138 を系統誤差とした.
以上の結果をまとめると$\rho=0.662 \pm 0.022 (stat.) \pm 0.138 (syst.)$ である.

\begin{figure}[hbt]
\centering
\includegraphics[width=0.6\textwidth]{figure/hatano/rho_modify1.eps}
\caption{$\rho$ のフィッティングの様子.黒点で示したのが測定値,赤線がフィッティングした曲線.}
\label{hatano_fig:rho}
\end{figure}

\subsubsection{ミッシェルパラメータ$\xi$ の導出}
式\eqref{eq:theory_michel} をエネルギーで積分すると,式\eqref{hatano_eq:xi} となる.これはスピンの向きに対する計数率の変化を表す.

\begin{equation}
  \frac{dN}{d\cos\theta} \propto \int^1_0 x^2dx\left[ (3-3x) + \frac{2}{3}\rho(4x-3) + \xi\cos\theta\left\{ (1-x) + \frac{2}{3}\delta(4x-3) \right\}\right] = \frac{1}{4} + \frac{1}{12}\xi\cos\theta
  \label{hatano_eq:xi}
\end{equation}

計数についてスピンが正偏極($0\leq\theta\leq\frac{\pi}{2}$) の時を$N_+$ ,負偏極($\frac{\pi}{2}\leq\theta\leq\pi$) の時を$N_-$ とし計数の比を$R=\frac{N_+}{N_-}$ と定義する.
式\eqref{hatano_eq:xi} を$\theta$について積分すると式\eqref{hatano_eq:xi_plus},\eqref{hatano_eq:xi_minus},\eqref{hatano_eq:xi_R} となる.
\begin{eqnarray}
  N_+ & \propto & \int^0_1 d(\cos\theta) \left(\frac{1}{4} + \frac{1}{12}\xi\cos\theta\right)=-\frac{1}{4}-\frac{1}{24}\xi \label{hatano_eq:xi_plus} \\
  N_- & \propto & \int^{-1}_0 d(\cos\theta) \left(\frac{1}{4} + \frac{1}{12}\xi\cos\theta\right)=-\frac{1}{4}+\frac{1}{24}\xi \label{hatano_eq:xi_minus} \\
  R  & = & \frac{6+\xi}{6-\xi} \label{hatano_eq:xi_R}
\end{eqnarray}

これはスピンの向きに関する部分の測定なので,単純なバックグラウンドだけでなくターゲット内散乱により無偏極となったものによるバックグラウンドも考えなければならない.
特に後者の影響を見積もる事は難しいが,どちらも$\xi$ を小さく見積もる方向に影響を与える.

最初の一周期の該当する範囲で重みを付けた計数は$N_+=73607.5$,$N_-=52878.6$ となる.
式\label{hatano_eq:xi_ratio} を$\xi$ について解くと$\xi=6(R-1) / (R+1)$ となり,これより$\xi=0.983\pm0.017 (stat.)$ となる.

\subsubsection{ミッシェルパラメータ$\delta$ の導出}
式\eqref{eq:theory_michel} よりスピンが正偏極の時から負偏極の時のエネルギースペクトラムを引くと,式\eqref{hatano_eq:delta} となる.ただし,この時スピンの偏極の割合は同じものとする.
この式よりミッシェルパラメータ$\delta$ を決定することができる.
\begin{equation}
  \frac{d\Gamma}{x^2dx} \propto 2 \int d(\cos\theta) \xi \left[ (1-x) + \frac{2}{3}\delta(4x-3) \right] \propto (1-x) + \frac{2}{3}\delta(4x-3)
  \label{hatano_eq:delta}
\end{equation}


正偏極のデータは$0\leq\theta\leq\frac{\pi}{2}$ を負偏極のデータは$\frac{\pi}{2}\leq\theta\leq\pi$ の最初の二周期の該当する範囲で重みをつけたエネルギースペクトラムを用いた.
正偏極のデータから負偏極のデータを差し引いたエネルギースペクトルに対して$\rho$ の解析と同様にフィッティングを行った.
その様子が\figref{hatano_fig:delta} である.
フィッティングのパラメータは\eqref{hatano_eq:delta} を$p_0(1-x/E_{max})+p_1\frac{2}{3}(4x/E_{max}-3)$ ,バックグラウンドを$p_2+p_3x$ と書くと\tabref{hatano_tab:delta} となった.
これよりミッシェルパラメータ$\delta$ は$\delta=0.641\pm0.117$ と求まった.
系統誤差も$\rho$ と同様に行うと最小で0.464,最大で0.794 となったので,大きい方の差の0.197 を系統誤差とした.
以上の結果より$\delta=0.641\pm0.117 (stat.) \pm0.197 (syst.)$ である.

\begin{table}[hbt]
\centering
\caption{$\delta$のフィッティングパラメータ}
\begin{tabular}{cccccccc}
$p_0$ & $\delta p_0$ & $p_1$ & $\delta p_1$ & $p_2$ & $\delta p_2$ & $p_3$ & $\delta p_3$ \\ \hline
0.978 & 0.123 & 1.595 & 0.211 & 104.741 & 2.722 & 0.211 & 0.123 \\
\end{tabular}
\label{hatano_tab:delta}
\end{table}

\begin{figure}[hbt]
\centering
\includegraphics[width=0.6\textwidth]{figure/hatano/delta_modify1.eps}
\caption{$\delta$ のフィッティングの様子.黒点で示したのが測定値,青線がフィッティングした曲線.}
\label{hatano_fig:delta}
\end{figure}

%\end{document}

\newpage
%Appendix
\appendix
%\documentclass[]{jsarticle}

%\usepackage[dvipdfmx]{graphicx}
%\usepackage{mathtools}
%\usepackage{cancel}
%\usepackage{amsmath,amssymb}
%\usepackage{cases}
%\usepackage{bm}

%\usepackage{feynmf}

%\newcommand{\Slash}[1]{{\ooalign{\hfil/\hfil\crcr\(#1\)}}}

%\begin{document}

%\appendix
\section{$\mu^{+}$ の崩壊寿命の理論}%\cite{}三野(一部小田川が修正)

\begin{align}
\mathcal{M} = &-g_{W}^2\left[\Bar{u}(\bm{q_{1}})\gamma^{\alpha}(1 - \gamma_{5})v(\bm{p'})\right]
\frac{-(-g_{\alpha\beta} + k_{\alpha}k_{\beta}/m_{W}^2)}{k^{2} - m_{W}^2 + i\epsilon}\left[\Bar{v}(\bm{p})\gamma^{\beta}(1 - \gamma_5)v(\bm{q_{2}})\right]
\end{align}
%
$m_{W}^2$ が$k^2$ にくらべて十分大きいとし,$m_{W} \rightarrow \infty$ の極限をとると,
%
\begin{align}
\mathcal{M} = &-\frac{iG}{\sqrt{2}}\left[\Bar{u}(\bm{q_{1}})\gamma^{\alpha}(1 - \gamma_{5})v(\bm{p'})\right]
\left[\Bar{v}(\bm{p})\gamma_{\alpha}(1 - \gamma_5)v(\bm{q_{2}})\right]
\end{align}
%
ミューオンの微分崩壊幅は次の式から得られる.
%
\begin{align}
d\Gamma = (2\pi)^{4}\delta^{(4)}(p'+q_{1}+q_{2}-p)\frac{m_{\mu}m_{e}m_{\nu_{e}}m_{\nu_{\mu}}}{E}\times\frac{1}{(2\pi)^9}\frac{d^3\bm{p'}}{E'}
\frac{d^3\bm{q_{1}}}{E_{1}}\frac{d^3\bm{q_{2}}}{E_{2}}|\mathcal{M}|^{2}\label{eq:decay}
\end{align}
%
全崩壊幅を計算するために始状態のスピンについて平均をとり、終状態のスピンについて和をとる.
\[\Gamma^{\alpha}=\gamma^{\alpha}(1 - \gamma_{5})\]
%
とおき、エネルギー射影演算子
$\begin{dcases}
\Delta^{+}_{\alpha\beta}(\bm{p})=\sum_{r=1}^{2}u_{r\alpha}(\bm{p})\bar{u}_{r\beta}(\bm{p})=\left(\frac{\Slash{p}+m}{2m}\right)_{\alpha\beta}\\
\Delta^{-}_{\alpha\beta}(\bm{p})=-\sum_{r=1}^{2}v_{r\alpha}(\bm{p})\bar{v}_{r\beta}(\bm{p})=\left(\frac{-\Slash{p}+m}{2m}\right)_{\alpha\beta}
\end{dcases}$
を用いると
%
\begin{align}
m_{\mu}m_{e}m_{\nu_{e}}m_{\nu_{\mu}}\frac{1}{2}\sum_\mathrm{spins}|\mathcal{M}|^{2}
&=m_{\mu}m_{e}m_{\nu_{e}}m_{\nu_{\mu}}\frac{G^{2}}{4}\sum_\mathrm{spins}\left[\Bar{u}(\bm{q_{1}})\Gamma^{\alpha}v(\bm{p'})\right]\left[\Bar{v}(\bm{p})\Gamma_{\alpha}v(\bm{q_{2}})\right]
\left[\Bar{v}(\bm{q_{2}})\Gamma_{\beta}v(\bm{p})\right]\left[\Bar{v}(\bm{p'})\Gamma^{\beta}u(\bm{q_{1}})\right]\notag\\
&=m_{\mu}m_{e}m_{\nu_{e}}m_{\nu_{\mu}}\frac{G^{2}}{4}\sum_{r=1}^2\sum_{r'=1}^2\sum_{r_1=1}^2\sum_{r_2=1}^2\notag\\
&\qquad\left[\Bar{u}_{r_{1}a}(\bm{q_{1}})\Gamma_{ab}^{\alpha}v_{r'b}(\bm{p'})\right]\left[\Bar{v}_{rc}(\bm{p})\Gamma_{\alpha cd}v_{r_{2}d}(\bm{q_{2}})\right]
\left[\Bar{v}_{r_{2}e}(\bm{q_{2}})\Gamma_{\beta ef}v_{rf}(\bm{p})\right]\left[\Bar{v}_{r'g}(\bm{p'})\Gamma_{gh}^{\beta}u_{r_{1}h}(\bm{q_{1}})\right]\notag\\
&=m_{\mu}m_{e}m_{\nu_{e}}m_{\nu_{\mu}}\frac{G^{2}}{4}\mathrm{Tr}\left[\frac{\Slash{q_{1}}+m_{\nu_{e}}}{2m_{\nu_{e}}}\Gamma^{\alpha}\frac{\Slash{p'}-m_{e}}{2m_{e}}\Gamma^{\beta}\right]
\mathrm{Tr}\left[\frac{\Slash{p}-m_{\mu}}{2m_{\mu}}\Gamma_{\alpha}\frac{\Slash{q_{2}}-m_{\nu_{\mu}}}{2m_{\nu_{\mu}}}\Gamma_{\beta}\right]\notag\\
&=\frac{G^{2}}{64}\mathrm{Tr}\left[\Slash{q_{1}}\Gamma^{\alpha}\Slash{p'}\Gamma^{\beta}\right]\mathrm{Tr}\left[\Slash{p}\Gamma_{\alpha}\Slash{q_{2}}\Gamma_{\beta}\right]\label{eq:spins}
\end{align}
%
最後の式変形では$m_{\nu_{e}}\rightarrow 0$,$m_{\nu_{\mu}}\rightarrow 0$ の極限をとり、
奇数個の$\gamma$ 行列の積のトレースは0 であることを用いた.
まずは式(\ref{eq:spins}) の最初のトレースを計算する.
%
\begin{align}
E^{\alpha\beta}\equiv \mathrm{Tr}\left[\Slash{q_{1}}\gamma^{\alpha}(1 - \gamma_{5})\Slash{p'}\gamma^{\beta}(1 - \gamma_{5})\right]
\end{align}
%
と定義し、以下の関係式を用いると
%
\begin{align*}
  \left\{
    \begin{array}{l}
      \left\{\gamma_{5},\gamma^{\alpha}\right\}=0\;(\alpha=0,\dots,3) \\
      (1-\gamma_{5})^{2}=2(1-\gamma_{5}) \\
      \mathrm{Tr}(\gamma^{\alpha}\gamma^{\beta}\gamma^{\gamma}\gamma^{\delta})=4(g^{\alpha\beta}g^{\gamma\delta}-g^{\alpha\gamma}g^{\beta\delta}+g^{\alpha\delta}g^{\beta\gamma}) \\
      \mathrm{Tr}(\gamma_{5}\gamma^{\alpha}\gamma^{\beta}\gamma^{\gamma}\gamma^{\delta})=-4i\epsilon^{\alpha\beta\gamma\delta}
    \end{array}
  \right.
\end{align*}
\begin{align}
E^{\alpha\beta}&=2q_{1\mu}p'_{\nu}\mathrm{Tr}\left[\gamma^{\mu}\gamma^{\alpha}\gamma^{\nu}\gamma^{\beta}(1-\gamma_{5})\right]\notag\\
&=8q_{1\mu}p'_{\nu}(g^{\mu\alpha}g^{\nu\beta}-g^{\mu\nu}g^{\alpha\beta}+g^{\mu\beta}g^{\alpha\nu}+i\epsilon^{\mu\alpha\nu\beta})\notag\\
&=8q_{1\mu}p'_{\nu}x^{\mu\alpha\nu\beta}\label{eq:E}
\end{align}
%
ここで
\[x^{\mu\alpha\nu\beta}\equiv g^{\mu\alpha}g^{\nu\beta}-g^{\mu\nu}g^{\alpha\beta}+g^{\mu\beta}g^{\alpha\nu}+i\epsilon^{\mu\alpha\nu\beta}\]
と定義した.式(\ref{eq:spins}) の2 つ目のトレースも同様に計算すると、
%
\begin{align}
  M_{\alpha\beta}&\equiv \mathrm{Tr}\left[\Slash{p}\gamma_{\alpha}(1 - \gamma_{5})\Slash{q_{2}}\gamma_{\beta}(1 - \gamma_{5})\right]\notag\\
  &= 8p^{\sigma}q_{2}^{\tau}x_{\sigma\alpha\tau\beta}\label{eq:M}
\end{align}
%
が得られる.
%
\begin{align}
  x^{\mu\alpha\nu\beta}x_{\sigma\alpha\tau\beta}&=(g^{\mu\alpha}g^{\nu\beta}-g^{\mu\nu}g^{\alpha\beta}+g^{\mu\beta}g^{\alpha\nu}+i\epsilon^{\mu\alpha\nu\beta})\notag\\
  &\qquad\qquad\times(g_{\sigma\alpha}g_{\tau\beta}-g_{\sigma\tau}g_{\alpha\beta}+g_{\sigma\beta}g_{\alpha\tau}+i\epsilon_{\sigma\alpha\tau\beta})\notag\\
  &=g^{\mu}_{\sigma}g^{\nu}_{\tau}-\xcancel{g^{\mu\nu}g_{\sigma\tau}}+g^{\mu}_{\tau}g^{\nu}_{\sigma}\notag\\
  &\qquad\xcancel{-g^{\mu\nu}g_{\sigma\tau}}+\xcancel{4g^{\mu\nu}g_{\sigma\tau}}-\xcancel{g^{\mu\nu}g_{\sigma\tau}}\notag\\
  &\qquad\qquad +g^{\mu}_{\tau}g^{\nu}_{\sigma}-\xcancel{g^{\mu\nu}g_{\sigma\tau}}+g^{\mu}_{\sigma}g^{\nu}_{\tau}-\epsilon^{\mu\alpha\nu\beta}\epsilon_{\sigma\alpha\tau\beta}\notag\\
  (\epsilon^{\mu\alpha\nu\beta}\epsilon_{\sigma\alpha\tau\beta}=-2(g^{\mu}_{\tau}g^{\nu}_{\tau}-g^{\mu}_{\tau}g^{\nu}_{\sigma})より)
  &=2(g^{\mu}_{\sigma}g^{\nu}_{\tau}+g^{\mu}_{\tau}g^{\nu}_{\sigma})+2(g^{\mu}_{\sigma}g^{\nu}_{\tau}-g^{\mu}_{\tau}g^{\nu}_{\sigma})\notag\\
  &=4g^{\mu}_{\sigma}g^{\nu}_{\tau}\label{eq:x}
\end{align}
%
式(\ref{eq:E})(\ref{eq:M})(\ref{eq:x}) を用いると、
%
\begin{align}
  m_{\mu}m_{e}m_{\nu_{e}}m_{\nu_{\mu}}\frac{1}{2}\sum_\mathrm{spins}|\mathcal{M}|^{2}
  &=G^{2}q_{1\mu}p'_{\nu}x^{\mu\alpha\nu\beta}p^{\sigma}q_{2}^{\tau}x_{\sigma\alpha\tau\beta}\notag\\
  &=4G^{2}(q_{1}p)(p'q_{2})\label{eq:spins_new}
\end{align}
%
式(\ref{eq:decay}) と式(\ref{eq:spins_new}) を組み合わせると微分崩壊幅は
%
\begin{align}
  d\Gamma = \frac{4G^{2}}{(2\pi)^{5}E}(q_{1}p)(p'q_{2})\delta^{(4)}(p'+q_{1}+q_{2}-p)\frac{d^3\bm{p'}}{E'}
  \frac{d^3\bm{q_{1}}}{E_{1}}\frac{d^3\bm{q_{2}}}{E_{2}}
\end{align}
%
次に2 つのニュートリノの運動量に関する積分を行う.
%
\begin{align}
d\Gamma&=\frac{4G^{2}}{(2\pi)^{5}E}\frac{d^{3}\bm{p'}}{E'}p_{\mu}p'_{\nu}\int d^{3}\bm{q_{1}}d^{3}\bm{q_{2}}\frac{q_{1}^{\mu}q_{2}^{\nu}}{E_{1}E_{2}}\delta^{(4)}(q_{1}+q_{2}-p+p')\notag\\
&=\frac{4G^{2}}{(2\pi)^{5}E}\frac{d^{3}\bm{p'}}{E'}p_{\mu}p'_{\nu}I^{\mu\nu}(q)\label{eq:decay_new}
\end{align}
%
\[\begin{dcases}
    I^{\mu\nu}(q)\equiv\int d^{3}\bm{q_{1}}d^{3}\bm{q_{2}}\frac{q_{1}^{\mu}q_{2}^{\nu}}{E_{1}E_{2}}\delta^{(4)}(q_{1}+q_{2}-q)\\
    q\equiv p-p'
\end{dcases}\]
%
$I^{\mu\nu}(q)$ のローレンツ共変性より、一般的な形は
\[I^{\mu\nu}(q)=g^{\mu\nu}A(q^{2})+q^{\mu}q^{\nu}B(q^{2})\]
と表せるので
%
\begin{subnumcases}
{}
g_{\mu\nu}I^{\mu\nu}(q)=4A(q^{2})+q^{2}B(q^{2})\label{eq:sub1}\\
q_{\mu}q_{\nu}I^{\mu\nu}(q)=q^{2}A(q^{2})+(q^{2})^{2}B(q^{2})\label{eq:sub2}
\end{subnumcases}
%
これ以降、ニュートリノの質量を0 とするので、$q_{1}^{2}=q_{2}^{2}=0$ となる.$I^{\mu\nu}(q)$ の$\delta$ 関数より、
\begin{align}
  q&=q_{1}+q_{2}\notag\\
  \Rightarrow q^{2}&=q_{1}^{2}+2(q_{1}q_{2})+q_{2}^{2}\notag\\
  \Rightarrow q^{2}&=2(q_{1}q_{2})\label{eq:momentum}
\end{align}
%
$A(q^{2}) とB(q^{2})$ の形を求めるために式(\ref{eq:sub1}) と式(\ref{eq:sub2}) の左辺の形を計算する.
%
\begin{align}
  g_{\mu\nu}I^{\mu\nu}(q)&=\int d^{3}\bm{q_{1}}d^{3}\bm{q_{2}}\frac{(q_{1}q_{2})}{E_{1}E_{2}}\delta^{(4)}(q_{1}+q_{2}-q)\notag\\
  &=\frac{q^{2}}{2}\int\frac{d^{3}\bm{q_{1}}}{E_{1}}\frac{d^{3}\bm{q_{2}}}{E_{2}}\delta^{(4)}(q_{1}+q_{2}-q)\qquad(\because(\ref{eq:momentum}))\notag\\
  &=\frac{q^{2}}{2}I(q^{2})\label{eq:sub3}
\end{align}
%
\[I(q^{2})\equiv\int\frac{d^{3}\bm{q_{1}}}{E_{1}}\frac{d^{3}\bm{q_{2}}}{E_{2}}\delta^{(4)}(q_{1}+q_{2}-q)\]
%
定義より$I(q^{2})$ は不変量なので、どの座標系をとってもよい.2 つのニュートリノの重心系を選ぶと、
\[\bm{q_{1}}=-\bm{q_{2}}\Leftrightarrow\bm{q}=0\]
ニュートリノのエネルギー$\omega$ はともに
\[\omega\equiv E_{1}=|\bm{q_{1}}|=E_{2}=|\bm{q_{2}}|\]
したがって
%
\begin{align}
  I(q^{2})&=\int\frac{d^{3}\bm{q_{1}}}{E_{1}}\frac{d^{3}\bm{q_{2}}}{E_{2}}\delta(E_{1}+E_{2}-q_{0})\delta^{(3)}(\bm{q_{1}}+\bm{q_{2}}-\bm{q})\notag\\
  &=\int d^{3}\bm{q_{1}}\frac{\delta(2\omega-q_{0})}{\omega^{2}}\notag\\
  &=4\pi\int d\omega\delta(2\omega-q_{0})\notag\\
  &=2\pi\int d\omega\delta(\omega-\frac{q_{0}}{2})\qquad(\because\delta(ax)=\frac{1}{|a|}\delta(x))\notag\\
  &=2\pi
\end{align}
%
式(\ref{eq:sub3}) より
%
\begin{align}
g_{\mu\nu}I^{\mu\nu}(q)=\pi q^{2}\label{eq:sub4}
\end{align}
%
同様に式(\ref{eq:sub2}) の左辺を
%
\[\begin{cases}
  qq_{1}=(q_{1}+q_{2})q_{1}=q_{1}q_{2}=\frac{q^{2}}{2}\quad(\because q_{1}^{2}=0)\\
  qq_{2}=(q_{1}+q_{2})q_{2}=q_{1}q_{2}=\frac{q^{2}}{2}\quad(\because q_{2}^{2}=0)
\end{cases}\]
%
を用いて計算する.
%
\begin{align}
q_{\mu}q_{\nu}I^{\mu\nu}(q)&=\int\frac{d^{3}\bm{q_{1}}}{E_{1}}\frac{d^{3}\bm{q_{2}}}{E_{2}}(qq_{1})(qq_{2})\delta^{(4)}(q_{1}+q_{2}-q)\notag\\
&=\left(\frac{q^{2}}{2}\right)^{2}I(q^{2})\notag\\
&=\frac{\pi}{2}(q^{2})^{2}\label{eq:sub5}
\end{align}
%
式(\ref{eq:sub4}) と式(\ref{eq:sub5}) をまとめると
%
\begin{subnumcases}
{}
g_{\mu\nu}I^{\mu\nu}(q)=4A(q^{2})+q^{2}B(q^{2})=\pi q^{2}\label{eq:sub6}\\
q_{\mu}q_{\nu}I^{\mu\nu}(q)=q^{2}A(q^{2})+(q^{2})^{2}B(q^{2})=\frac{\pi}{2}(q^{2})^{2}\label{eq:sub7}
\end{subnumcases}
%
$(\ref{eq:sub6})\times q^{2}-(\ref{eq:sub7})$ より
\[3q^{2}A(q^{2})=\frac{\pi}{2}(q^{2})^{2}\Rightarrow A(q^{2})=\frac{\pi}{6}q^{2}\]
(\ref{eq:sub6}) に代入すると
\[q^{2}B(q^{2})=\frac{\pi}{3}q^{2}\Rightarrow B(q^{2})=\frac{\pi}{3}\]
したがって、
\begin{align}
  I^{\mu\nu}(q)=\frac{\pi}{6}(g^{\mu\nu}q^{2}+2q^{\mu}q^{\nu})\label{eq:sub8}
\end{align}
%
式(\ref{eq:decay_new}) に式(\ref{eq:sub8}) を代入すると微分崩壊幅は、
%
\begin{align}
  d\Gamma&=\frac{4G^{2}}{(2\pi)^{5}E}\frac{d^{3}\bm{p'}}{E'}p_{\mu}p'_{\nu}\times \frac{\pi}{6}(g^{\mu\nu}q^{2}+2q^{\mu}q^{\nu})\notag\\
  &=\frac{2\pi}{3}\frac{G^{2}}{(2\pi)^{5}E}\frac{d^{3}\bm{p'}}{E'}[(pp')+2(pq)(p'q)]
\end{align}
%
最後に陽電子の運動量$\bm{p'}$ について積分する.ミューオンの静止系では
\begin{align*}
  &\begin{cases}
    p=(m_{\mu},0)\\
    q=p-p'
  \end{cases}
  \Rightarrow\quad
  \begin{cases}
    q_{0}=m_{\mu}-E'\\
    \bm{q}=-\bm{p'}
  \end{cases}\\
  &\begin{cases}
    pp'=m_{\mu}E'\\
    q^{2}=p^{2}-2pp'+p'^{2}=m_{\mu}^{2}-2m_{\mu}E'+m_{e}^{2}\\
    pq=m_{\mu}q_{0}=m_{\mu}(m_{\mu}-E')\\
    p'q=E'q_{0}-\bm{p'\cdot q}=E'(m_{\mu}-E')+|\bm{p'}|^{2}=m_{\mu}E'-m_{e}^{2}
  \end{cases}
\end{align*}
%
\begin{align}
  d\Gamma&=\frac{2\pi}{3}\frac{G^{2}}{(2\pi)^{5}m_{\mu}}\frac{d^{3}\bm{p'}}{E'}[m_{\mu}E'(m_{\mu}^{2}-2m_{\mu}E'+m_{e}^{2})+2m_{\mu}(m_{\mu}-E')(m_{\mu}E'-m_{e}^{2})]\notag\\
  &=\frac{2\pi}{3}\frac{G^{2}}{(2\pi)^{5}m_{\mu}}|\bm{p'}|dE'd\Omega'[m_{\mu}E'(m_{\mu}^{2}-2m_{\mu}E'+m_{e}^{2})+2m_{\mu}(m_{\mu}-E')(m_{\mu}E'-m_{e}^{2})]
\end{align}
%
ここで
\begin{align*}
  E'^{2}=m_{e}^{2}+|\bm{p'}|^{2}\Rightarrow E'dE'=|\bm{p'}|d|\bm{p'}|\\
  \therefore d^{3}\bm{p'}=|\bm{p'}|^{2}d|\bm{p'}|d\Omega'=|\bm{p'}|E'dE'd\Omega'
\end{align*}
となることを用いた.$\frac{m_{e}^{2}}{m_{\mu}^{2}}$ のオーダーの項を無視すると、
%
\begin{align}
  d\Gamma&=\frac{2\pi}{3}\frac{G^{2}}{(2\pi)^{5}}\frac{\sqrt{E'^{2}-\xcancel{m_{e}^{2}}}}{m_{\mu}}dE'd\Omega'[m_{\mu}E'(m_{\mu}^{2}-2m_{\mu}E'+\xcancel{m_{e}^{2}})+2m_{\mu}(m_{\mu}-E')(m_{\mu}E'-\xcancel{m_{e}^{2}})]\notag\\
  &\approx\frac{2\pi}{3}\frac{G^{2}}{(2\pi)^{5}}m_{\mu}E'^{2}dE'd\Omega'(3m_{\mu}-4E')
\end{align}
%
$E'$ について積分するために$\mu^{+}$ の崩壊によって放出される$e^{+}$ のエネルギー$E'$ の範囲を考える.ミューオンの静止系では
%
\[\begin{cases}
  p=(m_{\mu},0)\\
  p'=(E',\bm{p'})\\
  q_{1}=(E_{1},\bm{q_{1}})\\
  q_{2}=(E_{2},\bm{q_{2}})\\
  q_{1}^{2}=0\Leftrightarrow E_{1}=|\bm{q_{1}}|\\
  q_{2}^{2}=0\Leftrightarrow E_{2}=|\bm{q_{2}}|
\end{cases}\]
%
エネルギー・運動量保存則より
%
\begin{align*}
  &\qquad p=p'+q_{1}+q_{2}\\
  &\Leftrightarrow (p-p')^{2}=(q_{1}+q_{2})^{2}\\
  &\Leftrightarrow m_{\mu}^{2}-2m_{\mu}E'+m_{e}^{2}=2(E_{1}E_{2}-|\bm{q_{1}}||\bm{q_{2}}|cos\varphi)\quad(ただし\bm{q_{1}} と\bm{q_{2}} のなす角を\varphi とした)\\
  &\Leftrightarrow E'=\frac{m_{\mu}^{2}+m_{e}^{2}-2E_{1}E_{2}(1-cos\varphi)}{2m_{\mu}}
  \le \frac{m_{\mu}^{2}+m_{e}^{2}}{2m_{\mu}}
\end{align*}
%
電子の質量を無視すると$E'$ の範囲は、
%
\[
0\le E'\le \frac{1}{2}m_{\mu}
\]
%
全立体角と$E'$ ($0\le E'\le\frac{1}{2}m_{\mu}$) について積分すると、
\begin{align}
  \Gamma&=\frac{2\pi}{3}\frac{G^{2}}{(2\pi)^{5}}m_{\mu}\int_{0}^{4\pi}d\Omega'\int_{0}^{\frac{1}{2}m_{\mu}}dE'E'^{2}(3m_{\mu}-4E')\notag\\
  &=\frac{8\pi^{2}}{3}\frac{G^{2}}{(2\pi)^{5}}m_{\mu}\left[m_{\mu}E'^{3}-E'^{4}\right]_{0}^{\frac{1}{2}m_{\mu}}\notag\\
  &=\frac{2}{3}\frac{G^{2}}{(2\pi)^{3}}m_{\mu}\times \frac{m_{\mu}^{4}}{16}\notag\\
  &=\frac{G^{2}m_{\mu}^{5}}{192\pi^{3}}
\end{align}
%
以上より$\mu^{+}$ の寿命$\tau_{\mu}$ は
\[\tau_{\mu}=\frac{1}{\Gamma}=\frac{192\pi^{3}}{G^{2}m_{\mu}^{5}}\]
%
%\section{the Michel parameters の理論}
%\section{Geant4 で実際に用いたシミュレーションコード}
%\section{FEMM で実際の用いたシミュレーションコード}
%\section{DAQ で用いたコード}
%\section{ROOT で実際に解析に用いたコード}
%
%\end{document}

%参考文献
\begin{thebibliography}{99} 
\bibitem{magnet} 加速器用超伝導磁石 - KEK (\url{http://accwww2.kek.jp/oho/OHO\%20text\%20archives\%202005-2011/OHO11\%20web\%20final/OHO11\%20ogitsu\%2020110906.pdf}) 
\bibitem{leo} W.R.Leo (1994) Techniques for Nuclear and Particle Physics Experiments:Springer-Verlag
\bibitem{Mandl} Franz Mandl and Graham Show (2010) Quantum Field Theory 2nd Edition:Wiley
\end{thebibliography} 

\end{document}
