\section{最小二乗法}

\subsection{最小二乗法の原理}
測定で得られた数値を適当な関数で近似する際の関数の係数の推定方法として最尤推定法に基づくの手法が最小二乗法である.

まず,最小二乗法は以下の仮定を元にして推定を行う.
N回の測定を行い,測定値$(x_i,y_i)\quad(i=1,\cdots,N)$の組を得たとする.この時に測定データが
\begin{equation}
  y_i = f(x_i) + \epsilon
\end{equation}
で表されるとする.
$\epsilon$で表されるのが誤差の項でこの誤差は平均が0の正規分布に従うとする.
また,この誤差の分散は既知とする.
ただし,分散$\sigma_i$に対して$x_i$の依存性はあっても良いとする.
以上の考えに基づき,関数の係数(以下,パラメータaとよぶ)を決めた時にその関数$f_a$によって得られた測定値となる確率を考える.
\begin{equation}
  P(a) = \prod_{i=1}^N \frac{1}{\sqrt{2\pi\sigma^2}}\exp\left(-\frac{(y_i-f_a(x_i))^2}{2{\sigma_i}^2}\right)
\end{equation}
となる.

ここで最尤推定法に基づきこのパラメータaを推定することを考える.
すると$P(p)$(尤度関数)が最も大きくなるaが推定値(最尤推定値)である.
ここで尤度関数の対数をとると,
\begin{equation}
  log(P(a)) = C + \sigma_{i=1}^N \left(-\frac{(y_i-f_a(x_i))^2}{2{\sigma_i}^2}\right)
\end{equation}
となる.
ここで第一項は分散のみから決まり無視でき,第二項は負の値なのでここを$-\chi^2$とおく.
この時に尤度関数を最も大きくするときは$-\chi^2$を最小にした時である.
この測定値と最小値の二乗和を最小にするときをパラメータの推定値とするので最小二乗法と呼ばれる.
この$\chi^2$はパラメータの数をMと測定値をNとすると自由度(N-M)の$\chi^2$分布に従うことが知られている.
このパラメタの推定値はBLUE(最良線形不偏推定量)とよばれる性質を持つ.
また,以下で示すような関数の場合実際に計算も可能なためよく最小二乗法が用いられる.

\subsection{最小二乗法の計算}
このモデルとなる関数がM個のパラメータaに対して線形であると仮定する.
(非線形の場合は指数関数などの特別な場合を除き,反復解法によって近似値を求めることになる.)
すなわち,
\begin{equation}
  f(a) = \sigma_{j=1}^M a_j g_j(x)
\end{equation}
とかけるとする.この時に
\begin{equation}
  G_{ij} = g_j(x_i)
\end{equation}
とおけば,
\begin{equation}
  \chi^2 = \| Ga - y \|
  \label{chi_square_matrix_expression}
\end{equation}
と書ける.なので,これを最小にすれば良い.
これは正規方程式
\begin{equation}
  G^TGa=G^Ty
\end{equation}
を解くことにより,パラメタaが計算される.


\subsection{最小二乗法の応用}
今回の実験では,測定されるエネルギースペクトルに対してモデルとすべき関数は分かっている.
理論からエネルギースペクトラムがミッシェルパラメータで表される関数($michel(x)=\sum_{j=1}{M}a_j michel_j(x)$)で表現される.
ここで,ミッシェルパラメータaが線形でありこのパラメータを求めたい.
ただし,本論中に述べたようにこの関数でフィッティングをしてもうまく行かない.
それは実際に測定される値はここに線形な変換(電磁シャワー応答)がかかった値となるためである.
ここで,シミュレーションから計算される電磁シャワー応答を行列Mで表す.この行列は
\begin{eqnarray}
  E' = ME
\end{eqnarray}
という関係を表す.
具体的にEは本来のエネルギースペクトラムの値のベクトル,E'は観測されるエネルギースペクトラムのベクトルである.
今回の実験のフィッティングではこの電磁シャワーの応答を含めてモデル関数とした.

f(x)に今michel(x)をとると,式\eqref{chi_square_matrix_expression}のGaが本来のエネルギースペクトラムより予想される値を与える.
なので,これにMをかけてやれば,モデル関数に電磁シャワー応答を含めた事になる.
すなわち,
\begin{equation}
  \chi^2 = \| MGa - y \|
\end{equation}
とすれば良い.これは単に$G\rightarrow MG$という置き換えをしたに過ぎない.
そのため,同様に正規方程式を得ることができるので計算が可能となる.
